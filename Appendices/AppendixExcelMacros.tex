\chapter{Visual Basic for Applications For CPET Analysis} % Main appendix title
\label{AppendixExcelMacros} % For referencing this appendix elsewhere, use \ref{AppendixA}
\lhead{Appendix \ref{AppendixExcelMacros}. \emph{Excel VBA For CPET Analysis}} % This is for the header on each page - perhaps a shortened title
\clearpage
%------------------------------------------

\setstretch{1.0}
The following VBA code was written to facilitate detailed analysis of large volumes of raw CPET data that would not have been otherwise available from the final report generated for clinical use. This allowed analysis of all parameters measured at rest, anaerobic threshold and peak exercise rather than just $\dot{V}_{O_2}$.

\begin{lstlisting}
Option Explicit
Sub RenameSortSheets() 
   Dim ws As Worksheet
   Dim strName() As String
   For Each ws In Worksheets
     If ws.Range("$B$1").Value="" Then
       Exit Sub
     End If
     If ws.Name <> "aaa_main" Then
       strName=Split(CStr(ws.Range("$B$1").Value), "(")
       ws.Name=LCase(strName(0))
       ws.Activate
       ws.Range("B4").Select
       ActiveWindow.FreezePanes=True
     End If
   Next ws
   Dim N As Integer
   Dim M As Integer
   Dim FirstWSToSort As Integer
   Dim LastWSToSort As Integer
   Dim SortDescending As Boolean
   SortDescending=False
   If ActiveWindow.SelectedSheets.Count=1 Then
     'Change the 1 to the worksheet you want sorted first
     FirstWSToSort=1
     LastWSToSort=Worksheets.Count
    Else
     With ActiveWindow.SelectedSheets
       For N=2 To .Count
         If .Item(N - 1).Index <> .Item(N).Index - 1 Then
           MsgBox "You cannot sort non-adjacent sheets"
           Exit Sub
         End If
       Next N 
       FirstWSToSort=.Item(1).Index
       LastWSToSort=.Item(.Count).Index
     End With
   End If
   For M=FirstWSToSort To LastWSToSort
     For N=M To LastWSToSort
       If SortDescending=True Then
         If UCase(Worksheets(N).Name) > UCase(Worksheets(M).Name) Then
           Worksheets(N).Move Before:=Worksheets(M)
         End If
       Else
         If UCase(Worksheets(N).Name) < UCase(Worksheets(M).Name) Then
           Worksheets(N).Move Before:=Worksheets(M)
         End If
       End If
     Next N
   Next M
   Dim x As Integer
   x=2
   Sheets("aaa_main").Range("A:A").Clear
   Sheets("aaa_main").Cells(1, 1)="SheetName"
   For Each ws In Worksheets
     Sheets("aaa_main").Cells(x, 1)=ws.Name
     x=x + 1
   Next ws
End Sub
Sub GotoSheet()
  If ActiveWindow.ActiveSheet.Name="aaa_main" Then
    Dim selName As String
    selName=ActiveWindow.ActiveCell.Value
    Worksheets(selName).Activate
  Else
    Worksheets("aaa_main").Activate
  End If
End Sub
Sub VCO2VO2()
  Dim ws As Worksheet
  For Each ws In Worksheets
    Dim x As Integer
    x=4
    While ws.Cells(x, 2) > 0
      If ws.Cells(x, 9) <> "-" And ws.Cells(x, 6) <> "-" Then
        ws.Cells(x, 22)=CDbl(ws.Cells(x, 9)) / CDbl(ws.Cells(x, 6))
      End If
      x=x + 1
    Wend
  Next ws  
End Sub
Sub ClearColumn()
  Dim ws As Worksheet
  For Each ws In Worksheets
    ws.Range("AD9")=ws.Range("A1")
    ws.Range("A1").Clear
  Next ws
End Sub
Sub DeleteAllCharts()
  Dim ws As Worksheet
  Dim chart As ChartObject
  For Each ws In Worksheets
    For Each chart In ws.ChartObjects
      chart.Delete
    Next chart
  Next ws
End Sub
Sub AddRCValues()
  Dim x As Integer
  For x=3 To 88
    Sheets(Sheets("aaa_main").Cells(x, 1).Value).Range("AD8")=_
      Sheets("aaa_main").Cells(x, 27).Value
    Sheets(Sheets("aaa_main").Cells(x, 1).Value).Range("AD9")=_
      Sheets("aaa_main").Cells(x, 28).Value
  Next x
End Sub
Sub CopyLineForSPSS()
  Dim rng As Range
  Dim selrow As Integer
  Dim selrng As String  
  selrow=ActiveWindow.ActiveCell.Row
  selrng="AF" & CStr(selrow) & ":BX" & CStr(selrow)
  Range(selrng).Select
  Selection.Copy
End Sub

%----------------------------------
Sub ValuesAT()
  Dim r As Integer
  r=3 - Selection.Cells(1, 1).Row
  ActiveSheet.Range("X29")=Selection.Cells(r, 1)
  ActiveSheet.Range("Y29")=Selection.Cells(r, 3)
  ActiveSheet.Range("Z29")=Selection.Cells(r, 4)'VE
  ActiveSheet.Range("AA29")=Selection.Cells(r, 5)'VT
  ActiveSheet.Range("AB29")=Selection.Cells(r, 6)'VO2
  ActiveSheet.Range("AC29")=Selection.Cells(r, 7)'VO2/KG
  ActiveSheet.Range("AD29")=Selection.Cells(r, 8)'VE/VO2
  ActiveSheet.Range("AE29")=Selection.Cells(r, 9)'VCO2
  ActiveSheet.Range("AF29")=Selection.Cells(r, 10)'VE/VCO2
  ActiveSheet.Range("AG29")=Selection.Cells(r, 11)'RER
  ActiveSheet.Range("AH29")=Selection.Cells(r, 12)'PETO2
  ActiveSheet.Range("AI29")=Selection.Cells(r, 13)'PETCO2
  ActiveSheet.Range("AJ29")=Selection.Cells(r, 14)'O2PULSE
  ActiveSheet.Range("AK29")=Selection.Cells(r, 15)'HR
  ActiveSheet.Range("AL29")=Selection.Cells(r, 16)'Bf
  
  ActiveSheet.Range("W30")="AT"
  ActiveSheet.Range("X30")=WorksheetFunction.Average(Selection.Columns(1))'TIME
  ActiveSheet.Range("Y30")=WorksheetFunction.Average(Selection.Columns(3))'LOAD
  ActiveSheet.Range("Z30")=WorksheetFunction.Average(Selection.Columns(4))'VE
  ActiveSheet.Range("AA30")=WorksheetFunction.Average(Selection.Columns(5))'VT
  ActiveSheet.Range("AB30")=WorksheetFunction.Average(Selection.Columns(6))'VO2
  ActiveSheet.Range("AC30")=WorksheetFunction.Average(Selection.Columns(7))'VO2/KG
  ActiveSheet.Range("AD30")=WorksheetFunction.Average(Selection.Columns(8))'VE/VO2
  ActiveSheet.Range("AE30")=WorksheetFunction.Average(Selection.Columns(9))'VCO2
  ActiveSheet.Range("AF30")=WorksheetFunction.Average(Selection.Columns(10))'VE/VCO2
  ActiveSheet.Range("AG30")=WorksheetFunction.Average(Selection.Columns(11))'RER
  ActiveSheet.Range("AH30")=WorksheetFunction.Average(Selection.Columns(12))'PETO2
  ActiveSheet.Range("AI30")=WorksheetFunction.Average(Selection.Columns(13))'PETCO2
  ActiveSheet.Range("AJ30")=WorksheetFunction.Average(Selection.Columns(14))'O2PULSE
  ActiveSheet.Range("AK30")=WorksheetFunction.Average(Selection.Columns(15))'HR
  ActiveSheet.Range("AL30")=WorksheetFunction.Average(Selection.Columns(16))'Bf
End Sub

Sub ValuesPeak()
  ActiveSheet.Range("W31")="Peak"
  ActiveSheet.Range("X31")=WorksheetFunction.Average(Selection.Columns(1))'TIME
  ActiveSheet.Range("Y31")=WorksheetFunction.Max(Selection.Columns(3))'LOAD
  ActiveSheet.Range("Z31")=WorksheetFunction.Max(Selection.Columns(4))'VE
  ActiveSheet.Range("AA31")=WorksheetFunction.Max(Selection.Columns(5))'VT
  ActiveSheet.Range("AB31")=WorksheetFunction.Max(Selection.Columns(6))'VO2
  ActiveSheet.Range("AC31")=WorksheetFunction.Max(Selection.Columns(7))'VO2/KG
  ActiveSheet.Range("AD31")=WorksheetFunction.Max(Selection.Columns(8))'VE/VO2
  ActiveSheet.Range("AE31")=WorksheetFunction.Max(Selection.Columns(9))'VCO2
  ActiveSheet.Range("AF31")=WorksheetFunction.Max(Selection.Columns(10))'VE/VCO2
  ActiveSheet.Range("AG31")=WorksheetFunction.Max(Selection.Columns(11))'RER
  ActiveSheet.Range("AH31")=WorksheetFunction.Max(Selection.Columns(12))'PETO2
  ActiveSheet.Range("AI31")=WorksheetFunction.Max(Selection.Columns(13))'PETCO2
  ActiveSheet.Range("AJ31")=WorksheetFunction.Max(Selection.Columns(14))'O2PULSE
  ActiveSheet.Range("AK31")=WorksheetFunction.Max(Selection.Columns(15))'HR
  ActiveSheet.Range("AL31")=WorksheetFunction.Max(Selection.Columns(16))'Bf
End Sub

Sub ValuesOther()
  ActiveSheet.Range("W32")="Other"
  ActiveSheet.Range("X32")=WorksheetFunction.Average(Selection.Columns(1))'TIME
  ActiveSheet.Range("Y32")=WorksheetFunction.Average(Selection.Columns(3))'LOAD
  ActiveSheet.Range("Z32")=WorksheetFunction.Average(Selection.Columns(4))'VE
  ActiveSheet.Range("AA32")=WorksheetFunction.Average(Selection.Columns(5))'VT
  ActiveSheet.Range("AB32")=WorksheetFunction.Average(Selection.Columns(6))'VO2
  ActiveSheet.Range("AC32")=WorksheetFunction.Average(Selection.Columns(7))'VO2/KG
  ActiveSheet.Range("AD32")=WorksheetFunction.Average(Selection.Columns(8))'VE/VO2
  ActiveSheet.Range("AE32")=WorksheetFunction.Average(Selection.Columns(9))'VCO2
  ActiveSheet.Range("AF32")=WorksheetFunction.Average(Selection.Columns(10))'VE/VCO2
  ActiveSheet.Range("AG32")=WorksheetFunction.Average(Selection.Columns(11))'RER
  ActiveSheet.Range("AH32")=WorksheetFunction.Average(Selection.Columns(12))'PETO2
  ActiveSheet.Range("AI32")=WorksheetFunction.Average(Selection.Columns(13))'PETCO2
  ActiveSheet.Range("AJ32")=WorksheetFunction.Average(Selection.Columns(14))'O2PULSE
  ActiveSheet.Range("AK32")=WorksheetFunction.Average(Selection.Columns(15))'HR
  ActiveSheet.Range("AL32")=WorksheetFunction.Average(Selection.Columns(16))'Bf
End Sub

Sub AutomatePeakOtherValues()
  Dim ws As Worksheet
  Dim topRow As Integer
  Dim bottomRow As Integer
  For Each ws In Worksheets
    If ws.Name <> "aaa_main" Then
      ws.Activate
      'Fill other values
      Dim rng As Range
      Dim fdrng As Range
      Set rng=ws.Range("C:C")
      Set fdrng=rng.Find(30, LookIn:=xlValues)
      If Not fdrng Is Nothing Then
        topRow=fdrng(0, 1).Row
        bottomRow=fdrng(2, 0).Row
        Set rng=ws.Range(topRow & ":" & bottomRow)
        rng.Select
        Call ValuesOther
      End If
    
      'Fill peak values
      Dim maxVO2
      maxVO2=WorksheetFunction.Max(ws.Range("G:G"))
      Set rng=ws.Range("G:G")
      Set fdrng=rng.Find(maxVO2)
        If Not fdrng Is Nothing Then
          topRow=fdrng(-1, 1).Row
          bottomRow=fdrng(3, 0).Row
        If ws.Range("W31").Value="" Then
          Set rng=ws.Range(topRow & ":" & bottomRow)
          rng.Select
          Call ValuesPeak
        End If
      End If
      
    End If
  Next ws
End Sub

Sub ChartResize()
  Range("Y35").Select
  ActiveSheet.ChartObjects("Chart 1").Activate
  ActiveChart.Axes(xlCategory).Select
  ActiveChart.ChartArea.Select
  ActiveSheet.Shapes("Chart 1").ScaleWidth 0.5, msoFalse, _
  msoScaleFromBottomRight
  ActiveSheet.Shapes("Chart 1").IncrementLeft 102#
  ActiveSheet.Shapes("Chart 1").IncrementTop 253.5
End Sub

Sub VECharts()
  Dim ws As Worksheet
  For Each ws In Worksheets
    With ws.ChartObjects.Add _
      (Left:=800, Top:=725, Width:=400, Height:=225)
      .chart.ChartType=xlLine
      .chart.SetSourceData Source:=ws.Range("A:A,H:H,J:J") _
      , PlotBy:=xlColumns
      
      .chart.HasAxis(xlCategory, xlPrimary)=True
      .chart.HasAxis(xlValue, xlPrimary)=True
      
      .chart.Axes(xlCategory, xlPrimary).CategoryType=xlAutomatic
    End With
  Next ws
End Sub

Sub CopyCPETParametersFromSheetsToMain()
  Dim x As Integer
  For x=3 To 99
    Dim sheetname As String
    Sheets("aaa_main").Activate
    sheetname=ActiveSheet.Cells(x, 1)
    ActiveSheet.Cells(x, 32).Select
    Sheets(sheetname).Select
    Range("X30:AL30").Select
    Selection.Copy
    Sheets("aaa_main").Select
    ActiveSheet.Paste
    
    ActiveSheet.Cells(x, 47).Select
    Sheets(sheetname).Select
    Range("X31:AL31").Select
    Selection.Copy
    Sheets("aaa_main").Select
    ActiveSheet.Paste
    
    ActiveSheet.Cells(x, 62).Select
    Sheets(sheetname).Select
    Range("X32:AL32").Select
    Selection.Copy
    Sheets("aaa_main").Select
    ActiveSheet.Paste
  Next x
  
End Sub

Sub CopyVEVO2VEVCO2FromSheetsToMain()
  Dim x As Integer
  For x=3 To 99
    Dim sheetname As String
    Sheets("aaa_main").Activate
    sheetname=ActiveSheet.Cells(x, 1)
    ActiveSheet.Cells(x, 79).Select
    Sheets(sheetname).Select
    Range("Y22").Select
    Selection.Copy
    Sheets("aaa_main").Select
    ActiveSheet.Paste
    
    ActiveSheet.Cells(x, 80).Select
    Sheets(sheetname).Select
    Range("Y23").Select
    Selection.Copy
    Sheets("aaa_main").Select
    ActiveSheet.Paste
  Next x
  
End Sub

Sub FindStartEndTimeofExercise()
  Dim x As Integer
  Dim y As Integer
  Dim starttime As String
  Dim endtime As String
  Dim sheetname As String
  For x=3 To 99
    Sheets("aaa_main").Activate
    sheetname=ActiveSheet.Cells(x, 1)
    Sheets(sheetname).Select
    y=15
    While ActiveSheet.Cells(y, 3)="-"
      y=y + 1
    Wend
    starttime=ActiveSheet.Cells(y, 1)
    y=y + 5
    While ActiveSheet.Cells(y, 3) > "0"
      y=y + 1
    Wend
    endtime=ActiveSheet.Cells(y - 1, 1)
    Sheets("aaa_main").Activate
    ActiveSheet.Cells(x, 77)=starttime
    ActiveSheet.Cells(x, 78)=endtime
  Next x
End Sub

\end{lstlisting}

\setstretch{2.0}