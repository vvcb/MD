\chapter{Visual Basic for Applications For Data Collection}
\label{AppendixAccessDatabase}
\lhead{Appendix \ref{AppendixAccessDatabase}. \emph{MS Access VBA For Data Collection}}
\clearpage
%-------------------------------------------------------------         

\setstretch{1.0}

The following VBA code was written to facilitate collection of laboratory data from the existing user interface that did not allow easy automation. It forms part of a Microsoft Access database. Some parts of the code has been removed or reformatted to be accommodated within the constraints of this thesis.

The RealLink for Windows (RFW, http://www.nps-reality.com/reallink) application used at Glasgow Royal Infirmary allowed access form VBA via Dynamic Data Exchange. This was used to collect and parse laboratory results for preoperative and postoperative blood results efficiently and without error.

\begin{lstlisting}
'#########################################################################
'Dynamic Data Exchange with RealLink application
'#########################################################################
Option Compare Database
Function RLChannel() As Long
  Dim AppName As String
  Dim Topic As String
  AppName = "******"
  Topic = "******"
  DDETerminateAll
  
  On Error GoTo ErrorHandler
  RLChannel = DDEInitiate(AppName, Topic)
  Exit Function
  ErrorHandler:
  MsgBox "Error initiating connection to RealLink   Error ID: " & _
  Err.Description, vbOKOnly, "Error"
End Function
Sub RLSendData(strdata As String, WaitSecs As Integer)
  On Error GoTo ErrorHandler
  Dim starttime As Double
  AppActivate "RealLink   NGTBIO", Wait
  starttime = Timer
  SendKeys strdata, True
  starttime = Timer
  While Timer < starttime + WaitSecs: Wend
  Exit Sub
  ErrorHandler:
'Error gets dealt with by procedure calling this function
End Sub
  
Function RLGetData(ChannelNum As Long) As String
  On Error GoTo ErrorHandler
  Dim RLItem As String
  RLItem = "Table 0,0,80,24,Comma"
  RLGetData = DDERequest(ChannelNum, RLItem)
  Exit Function
  ErrorHandler:
  MsgBox "Error capturing data from RealLink   : " & _ 
    Err.Description, vbOKOnly, "Error"
End Function
  
Sub WaitFor(intSeconds As Integer)
  Dim starttime As Single
  starttime = Timer
  While Timer < starttime + intSeconds: Wend
End Sub
Function ChkLabScreen(strLabScreen As String, _
 strSurname As String, strLab As String) As Integer
  If InStr(1, strLabScreen, strSurname) = 0 Then 
    ChkLabScreen = ChkLabScreen + 1
  If InStr(1, strLabScreen, strLab) Then 
    ChkLabScreen = ChkLabScreen + 2
End Function
Sub RunRealLink()
'Wait for application to start then wait 5 seconds
'Shell runs commands asynchronously
'DDEPoke does not work   use windows shell and processID instead
  Dim dReturnValue As Single
  dReturnValue = Shell("C:\RFW\RFW.EXE /LC:\RFW\RFWDDE.DLL NGTBIO",_
   vbMaximizedFocus)
  While dReturnValue < 0: Wend
  Call WaitFor(5)
End Sub
Sub Login(LabNum As String, Username As String, Password As String)
' Login to Biochem/Haem
  Dim strSend As String
  
  strSend = "***" & vbCr
  Call RLSendData(strSend, 2)
  
  strSend = LabNum & vbCr
  Call RLSendData(strSend, 2)
  
  strSend = Username & vbCr
  Call RLSendData(strSend, 1)
  
  strSend = Password & vbCr
  Call RLSendData(strSend, 1)
End Sub
Sub SearchPatient(Surname As String, HospitalNumber As String, _
 Optional EarliestDate As Date, Optional LatestDate As Date)
'Enter patient information and date range(optional)
  Dim strSend As String
  strSend = ""
  Call RLSendData(strSend, 1)
  strSend = HospitalNumber & vbCr
  Call RLSendData(strSend, 1)
  strSend = Left(Surname, 2) & vbCr
  Call RLSendData(strSend, 1)
  strSend = Format(EarliestDate, "dd.mm.yy") & vbCr
  Call RLSendData(strSend, 0) 'Earliest date to search
  strSend = Format(LatestDate, "dd.mm.yy") & vbCr
  Call RLSendData(strSend, 0) 'Latest date to search
End Sub

'#########################################################  
'Parser for results scraped from RealLink screengrab
'#########################################################
Option Compare Database
'Results parsers
Public Sub ParseFBC(strHD As String)
On Error GoTo ErrorHandler
Dim strA() As String 'string to hold rows of data
Dim intRows As Long
Dim intTests As Integer
Dim intPosition As Integer
Dim strTest() As Variant 'name of test eg WBC
Dim strResult(10, 1) As String 
'change first dimension if changing number of tests
strTest = Array("WBC", "Hb", "Hct", "MCV", "Plts", _
"NEUT", "LYMPH", "MONO", "EOS", "BASO")
intTests = UBound(strTest)
strA = Split(strHD, vbCr)
intRows = UBound(strA)
Dim tempRow As String
Dim rowdate As Integer
For rowdate = 0 To 10
  tempRow = strA(rowdate)
  If InStr(1, tempRow, "Collected") > 0 Then
   Dim strDateTime() As String
   Dim strDate As String
   Dim strTime As String
   strDateTime = Split(tempRow, "Collected")
   strDateTime = Split(strDateTime(1), ",")
   If InStr(1, strDateTime(1), "*") > 0 Then
   strDate = Trim(Replace(strDateTime(1), ".", "/"))
   End If
   If UBound(strDateTime) > 2 Then
   strDate = Trim(Replace(strDateTime(1), ".", "/"))
   strTime = Trim(strDateTime(2))
   End If
  End If
Next rowdate
For t = 0 To intTests 'cycle through list of tests   strTest
  For i = 0 To intRows 'cycle through rows   strA
   tempRow = strA(i)
   tempRow = Replace(tempRow, vbCrLf, "")
   tempRow = Replace(tempRow, vbLf, "")
   tempRow = Replace(tempRow, vbCr, "")
   temptest = strTest(t)
   intPosition = InStr(tempRow, temptest)
   If intPosition > 0 Then
    Dim strTemp() As String
    strTemp = Split(tempRow, ",")
    
    For j = 0 To UBound(strTemp) 
    'cycle through values in a single row   strTemp
     If Trim(strTemp(j)) = strTest(t) Then
      strResult(t, 0) = strTest(t)
      
      If Trim(strTemp(j + 1)) = "+" Or Trim(strTemp(j + 1)) = " " Then
       strResult(t, 1) = Trim(strTemp(j + 2))
      Else
       strResult(t, 1) = Trim(strTemp(j + 1))
      End If
     End If
    Next j
   End If
  Next i
Next t
'Fill in results into form
  Call SaveHaem(strDate, strTime, strResult)
  Exit Sub
  ErrorHandler:
End Sub
Sub ParseCoag(strCoag As String)
  On Error GoTo ErrorHandler
  Dim strA() As String 'string to hold rows of data
  Dim intRows As Long
  Dim intTests As Integer
  Dim intPosition As Integer
  Dim strTest() As Variant 'name of test eg WBC
  Dim strResult(23, 1) As String 
  'change first dimension if changing number of tests
  strTest = Array("PT", "PT C Ratio", "APTT", "APTT Ratio", "TCT", "TCT Ratio")
  intTests = UBound(strTest)
  strA = Split(strCoag, vbCr)
  intRows = UBound(strA)
  Dim tempRow As String
  Dim rowdate As Integer
  For rowdate = 0 To 10
   tempRow = strA(rowdate)
   If InStr(1, tempRow, "Collected") > 0 Then
    Dim strDateTime() As String
    Dim strDate As String
    Dim strTime As String
    strDateTime = Split(tempRow, "Collected")
    strDateTime = Split(strDateTime(1), ",")
    If InStr(1, strDateTime(1), "*") > 0 Then
    strDate = Trim(Replace(strDateTime(1), ".", "/"))
    End If
    If UBound(strDateTime) > 2 Then
    strDate = Trim(Replace(strDateTime(1), ".", "/"))
    strTime = Trim(strDateTime(2))
    End If
   End If
  Next rowdate
  For t = 0 To intTests 'cycle through list of tests   strTest
   For i = 0 To intRows 'cycle through rows   strA
   tempRow = strA(i)
   tempRow = Replace(tempRow, ",+", "")
   tempRow = Replace(tempRow, ", ", "")
   tempRow = Replace(tempRow, "+,", ",")
   tempRow = Replace(tempRow, " ,", ",")
   tempRow = Replace(tempRow, "|", ",")
   temptest = strTest(t)
   intPosition = InStr(tempRow, temptest)
   If intPosition > 0 Then
    Dim strTemp() As String
    strTemp = Split(tempRow, ",")
    For j = 0 To UBound(strTemp) 
    'cycle through values in a single row   strTemp
     If Trim(strTemp(j)) = strTest(t) Then
      strResult(t, 0) = strTest(t)
      If Trim(strTemp(j + 1)) = "+" Or Trim(strTemp(j + 1)) = " " _
        Or Trim(strTemp(j + 1)) = "" Then
       strResult(t, 1) = Trim(strTemp(j + 2))
      Else
       strResult(t, 1) = Trim(strTemp(j + 1))
      End If
     End If
    Next j
   End If
    Next i
  Next t
'Fill in results into form
  Call SaveCoag(strDate, strTime, strResult)
  Exit Sub
  ErrorHandler:
End Sub
Sub ParseBio(strBD As String)
  Dim strSurname As String
  Dim strHNum As String
  Dim strA() As String 'string to hold rows of data
  Dim intRows As Long
  Dim intTests As Integer
  Dim intPosition As Integer
  Dim strTest() As Variant 'name of test eg Sodium
  strTest = Array("Sodium", "Potassium", "Urea", "Creat", "eGFR", "Phos",_
   "Mg++", "Chloride", "Bili", "G GT", "A Phos", "ALT", "AST",_
   "T Prot", "Alb", "Globulin", "Glucose", "CRP", "Trop I", _
   "Calcium", "Ad Cal", "Amylase")
  Dim strResult(21, 1) As String 
  'change first dimension if changing number of tests
  intTests = UBound(strTest)
  strBD = Replace(strBD, ".Amylase.", "..")
  strBD = Replace(strBD, vbCrLf, "")
  strBD = Replace(strBD, vbLf, "")
  strBD = Replace(strBD, vbCr, "")
  strA = Split(strBD, "|")
  Dim strTempDate As String
  strTempDate = strA(0)
  If InStr(strTempDate, "Collected") Then
   Dim strDateTime() As String
   Dim strDate As String
   Dim strTime As String
   strDateTime = Split(strTempDate, "Collected")
   strDateTime = Split(strDateTime(1), ",")
   strDate = strDateTime(1)
   strDate = Replace(strDate, ".", "/")
   strTime = strDateTime(2)
  End If
  intRows = UBound(strA)
  For t = 0 To intTests 'cycle through list of tests   strTest
   For i = 0 To intRows 'cycle through rows   strA
   tempRow = strA(i)
   temptest = strTest(t)
   intPosition = InStr(tempRow, temptest)
    If intPosition > 0 Then
     Dim strTemp() As String
     Dim intDots As Integer
     
    'Routine to remove dots word dots
     intDots = InStr(strA(i), "..")
     If intDots > 0 Then
      strTemp = Split(NoDots(strA(i)), ",")
     Else
      strTemp = Split(strA(i), ",")
     End If
    'End of Routine to remove dots word dots
     
     For j = 0 To UBound(strTemp) 
     'cycle through values in a single row   strTemp
      If Trim(strTemp(j)) = strTest(t) Then
       strResult(t, 0) = strTest(t)
       
       If Trim(strTemp(j + 1)) = "+" Or Trim(strTemp(j + 1)) = " " Then
        strResult(t, 1) = Trim(strTemp(j + 2))
       Else
        strResult(t, 1) = Trim(strTemp(j + 1))
       End If
      End If
     Next j
    End If
   Next i
  Next t
  'Fill in results into form
   Call SaveBio(strDate, strTime, strResult)
End Sub
Function NoDots(strWithDots As String) As String
  Dim intDotEndPos As Integer
  Dim intDotStartPos As Integer
  Dim intRowLen As Integer
  intDotEndPos = InStrRev(strWithDots, "..") 
  'find the first occurrence of .. from the end
  intDotStartPos = InStr(strWithDots, "..") 
  'find the first occurrence of .. from the end
  intRowLen = Len(strWithDots) 
  'get the total length of the string with dots
'get the substring on the right and left of the dots based on above values
  NoDots = Left(strWithDots, intDotStartPos) & "," & _
  Right(strWithDots, intRowLen   intDotEndPos   1)
'Debug.Print NoDots
End Function

'#########################################################
'Code to save results to respective tables
'#########################################################
Option Compare Database
Sub SaveBio(strDate, strTime, strResult)
Dim tf_B As SubForm
Set tf_B = Forms!frm_Demo!frm_Bloods
'create a new record
tf_B.SetFocus
DoCmd.GoToRecord , , acNewRec
tf_B!CHI = Forms!frm_Demo!CHI
tf_B!Surname = Forms!frm_Demo!Surname
tf_B!Date = strDate
tf_B!Time = strTime
intTests = UBound(strResult, 1)
For i = 0 To intTests
  Select Case strResult(i, 0)
  Case "Potassium"
   tf_B!K = strResult(i, 1)
  Case "Urea"
   tf_B!Urea = strResult(i, 1)
  Case "Creat"
   tf_B!Creatine = strResult(i, 1)
'Case "eGFR"
  Case "Phos"
   tf_B!PO4 = strResult(i, 1)
  Case "Mg++"
   tf_B!Magnesium = strResult(i, 1)
  Case "Chloride"
   tf_B!Cl = strResult(i, 1)
  Case "Bili"
   tf_B!Bilirubin = strResult(i, 1)
  Case "G GT"
   tf_B!GGT = strResult(i, 1)
  Case "A Phos"
   tf_B!Al_Phos = strResult(i, 1)
  Case "ALT"
   tf_B!ALT = strResult(i, 1)
  Case "AST"
   tf_B!AST = strResult(i, 1)
  Case "T Prot"
   tf_B!Protein = strResult(i, 1)
  Case "Alb"
   tf_B!Albumin = strResult(i, 1)
  Case "Globulin"
   tf_B!Globulin = strResult(i, 1)
  Case "Glucose"
   tf_B!Glucose = strResult(i, 1)
  Case "CRP"
   tf_B!CRP = strResult(i, 1)
  Case "Trop I"
   tf_B!Troponin = strResult(i, 1)
  Case "Calcium"
   tf_B!Ca = strResult(i, 1)
  Case "Ad Cal"
   tf_B!Adj_Ca = strResult(i, 1)
  Case "Amylase"
   tf_B!Amylase = strResult(i, 1)
  Case "Sodium"
   tf_B!Na = strResult(i, 1)
  End Select
  Next i
End Sub
Sub SaveCoag(TestDate As String, TestTime As String, Results() As String)
Dim tf_C As SubForm
Set tf_C = Forms!frm_Demo!frm_Coag
'create a new record
tf_C.SetFocus
DoCmd.GoToRecord , , acNewRec
tf_C!CHI = Forms!frm_Demo!CHI
tf_C!Surname = Forms!frm_Demo!Surname
tf_C!Date = TestDate
If TestTime <> "*" And TestTime <> "NK" Then tf_C!Time = TestTime
intTests = UBound(Results, 1)
For i = 0 To intTests
  Select Case Results(i, 0)
  Case "PT"
   tf_C!PT = Results(i, 1)
  Case "PT C Ratio"
   tf_C![PT C Ratio] = Results(i, 1)
  Case "APTT"
   tf_C!APTT = Results(i, 1)
  Case "APTT Ratio"
   tf_C![APTT Ratio] = Results(i, 1)
  Case "TCT"
   tf_C!TCT = Results(i, 1)
  Case "TCT Ratio"
   tf_C![TCT Ratio] = Results(i, 1)
  End Select
  Next i
End Sub
Sub SaveHaem(TestDate As String, TestTime As String, Results() As String)
Dim tf_H As SubForm
Set tf_H = Forms!frm_Demo!frm_Haematology
'create a new record
tf_H.SetFocus
DoCmd.GoToRecord , , acNewRec
tf_H!CHI = Forms!frm_Demo!CHI
tf_H!Surname = Forms!frm_Demo!Surname
tf_H!Date = TestDate
If TestTime <> "*" Then tf_H!Time = TestTime
intTests = UBound(Results, 1)
For i = 0 To intTests
'frmHaem.strResult(i, 0) = strResult(i, 1)
  Select Case Results(i, 0)
  Case "WBC"
   tf_H!WBC = Results(i, 1)
  Case "Plts"
   tf_H!Plts = Results(i, 1)
  Case "Hb"
   tf_H!HB = Results(i, 1)
  Case "MCV"
   tf_H!MCV = Results(i, 1)
  Case "MCHC"
   tf_H!MCHC = Results(i, 1)
  Case "RDW"
   tf_H!RDW = Results(i, 1)
  Case "RBC"
   tf_H!RBC = Results(i, 1)
  Case "NEUT"
   tf_H!NEUT = Results(i, 1)
  Case "LYMPH"
   tf_H!LYMPH = Results(i, 1)
  Case "MONO"
   tf_H!MONO = Results(i, 1)
  Case "EOS"
   tf_H!EOS = Results(i, 1)
  Case "BASO"
   tf_H!BASO = Results(i, 1)
  Case "Hct"
   tf_H!Hct = Results(i, 1)
  Case "MCH"
   tf_H!MCH = Results(i, 1)
  Case "MPV"
   tf_H!MPV = Results(i, 1)
  Case "PDW"
   tf_H!PDW = Results(i, 1)
  Case "PCT"
   tf_H!PCT = Results(i, 1)
  End Select
  Next i
End Sub

'#########################################################
'User interface functions
'#########################################################

Option Compare Database
Private Sub Form_Load()
  Me.RecordSource = "tb_Demo"
End Sub
Private Sub cmbAllPatients_AfterUpdate()
  ' Find the record that matches the control.
  Dim rs As Object
  Set rs = Me.Recordset.Clone
  rs.FindFirst "[IPID] = " & Str(Nz(Me![cmbAllPatients], 0))
  If Not rs.EOF Then Me.Bookmark = rs.Bookmark
End Sub
Private Sub cmbChooseProcedure_AfterUpdate()
  'Filter by operation name
  Dim strSQL As String
  Dim strFilter As String
  If Me.cmbChooseProcedure = "All" Or IsNull(Me.cmbChooseProcedure) Then
   ' If the combo is Null, use the whole table as the RecordSource.
   Me.RecordSource = "tb_Demo"
  Else
   strFilter = Me.cmbChooseProcedure
   strSQL = "SELECT tb_Demo.* FROM tb_Demo LEFT JOIN tb_Operation ON _
    tb_Demo.IPID = tb_Operation.IPID WHERE ((([tb_Operation]![Primary _
     Procedure]) LIKE '*" & strFilter & "*'))ORDER BY _
      [tb_Operation]![Operation Date];"
   Me.RecordSource = strSQL
  End If
End Sub
Private Sub cmbChooseOpDate_AfterUpdate()
  'Filter by operation DATE
  Dim strSQL As String
  Dim strFilter As Date
  If IsNull(Me.cmbChooseOpDate) Then
   ' If the combo is Null, use the whole table as the RecordSource.
   Me.RecordSource = "tb_Demo"
  Else
   strFilter = CDate(Me.cmbChooseOpDate)
   strSQL = "SELECT tb_Demo.* FROM tb_Demo LEFT JOIN tb_Operation ON _
    tb_Demo.IPID = tb_Operation.IPID WHERE _
    ((([tb_Operation]![Operation Date]) > #" & strFilter & "#))_
    ORDER BY [tb_Operation]![Operation Date];"
   Me.RecordSource = strSQL
  End If
End Sub
Private Sub cmbChooseSpecialty_AfterUpdate()
  'Filter by Specialty
  Dim strSQL As String
  Dim strFilter As String
  strFilter = Me.cmbChooseSpecialty
  If strFilter = "All" Or IsNull(Me.cmbChooseSpecialty) Then
   ' If the combo is Null, use the whole table as the RecordSource.
   Me.RecordSource = "tb_Demo"
  Else
   strSQL = "SELECT tb_Demo.* FROM tb_Demo LEFT JOIN tb_Event_Admission _
   ON tb_Demo.IPID = tb_Event_Admission.IPID WHERE _
    ((([tb_Event_Admission]![Specialty] _
    LIKE '*" & strFilter & "*')));"
   Me.RecordSource = strSQL
  End If
End Sub
Private Sub cmdNewPatient_Click()
  On Error GoTo Err_cmdNewPatient_Click
  DoCmd.DoMenuItem acFormBar, acRecordsMenu, 5, , acMenuVer70
  DoCmd.GoToRecord , , acNewRec
  Exit Sub
  Err_cmdNewPatient_Click:
  MsgBox Err.Description
End Sub
Private Sub listCurrentPatients_AfterUpdate()
  ' Find the record that matches the control.
  Dim rs As Object
  Set rs = Me.Recordset.Clone
  rs.FindFirst "[IPID] = " & Str(Nz(Me![listCurrentPatients], 0))
  If Not rs.EOF Then Me.Bookmark = rs.Bookmark
  Me.frm_Physiology.Form.Refresh
End Sub
Private Sub cmdSendDetails_Click()
  'checked and works
  Dim Surname As String 'Surname from frm_demo
  Dim HospitalNum As String 'Hospital number from frm_demo
  Dim OpDate As Date 'Date of operation from frm_operation
  Dim Edate As Date 'Earliest date of bloods to return
  Dim Ldate As Date 'Latest date of bloods to return
  On Error GoTo ErrorHandler:
  Surname = Me.Surname
  HospitalNum = Me.HospitalNum
  OpDate = Me.frm_Operation.Form![Operation Date]
  'Set search date range
  Edate = OpDate   10
  Ldate = OpDate + 5
  If Me.chkResuse = 0 Then 'Is Reuse box not checked then start a new RL session
   Call RunRealLink
   Select Case Me.frameChooseLab.Value
    Case 1
     Call Login("3", "******", "******")
    Case 2
     Call Login("2", "******", "******")
   End Select
  End If
  Call SearchPatient(Surname, HospitalNum, Edate, Ldate)
  Exit Sub
  ErrorHandler:
MsgBox "cmdHaem Error" & Err.Description
End Sub
Private Sub cmdImport_Click()
  On Error GoTo ErrorHandler
  Dim intRepeat As Integer
  KillSwitch = False
  If CInt(Me.txtAutoRepeat) > 0 Then
   intRepeat = CInt(Me.txtAutoRepeat)
  Else
   intRepeat = 1
  End If
  If Me.chkAutoUpdate.Value =  1 Then
   For i = 1 To intRepeat
    Call ImportData
    Me.Refresh
    Call cmdLater_Click
    If KillSwitch = True Then Exit Sub
   Next
  End If
  ErrorHandler:
  Exit Sub
End Sub
Private Sub ImportData()
  Dim LabChannel As Long
  LabChannel = RLChannel
  'Grab screen text from existing RL window
  Dim strResultScreen As String
  Call WaitFor(3)
  If LabChannel > 0 Then
   strResultScreen = RLGetData(LabChannel)
  Else
   MsgBox "RealLink connection could not be established", vbOKOnly,_
    "Connection Error"
   Exit Sub
  End If
  'Send to either FBC/Coag parser
  If InStr(1, strResultScreen, "Press Esc") Then KillSwitch = True
  If InStr(1, strResultScreen, "..FBC..") Then
   Call ParseFBC(strResultScreen)
  End If
  If InStr(1, strResultScreen, "..C/Screen..") Then
   Call ParseCoag(strResultScreen)
  End If
  If InStr(1, strResultScreen, "Biochem") Then
   Call ParseBio(strResultScreen)
  End If
  'Terminate all DDE connections
DDETerminate LabChannel
End Sub
Private Sub cmdEarlier_Click()
  Dim strTemp As String
  strTemp = "E" & vbCr
  Call RLSendData(strTemp, 0)
End Sub
Private Sub cmdLater_Click()
  Dim strTemp As String
  strTemp = "L" & vbCr
  Call RLSendData(strTemp, 0)
End Sub
Private Sub tb_ABG_ALL_subform_Enter()
  Me.Refresh
End Sub
Private Sub tgDemoDupes_Click()
  Select Case Me.tgDemoDupes.Value
   Case  1
    Me.tgDemoDupes.SetFocus
    Me.tgDemoDupes.Caption = "DemoDupes ON"
    Me.RecordSource = "qs_DemoDupes"
   Case 0
    Me.tgDemoDupes.SetFocus
    Me.tgDemoDupes.Caption = "DemoDupes OFF"
    Me.RecordSource = "tb_Demo"
  End Select
End Sub
Private Sub cmdRefresh_Click()
  On Error GoTo Err_cmdRefresh_Click
  DoCmd.DoMenuItem acFormBar, acRecordsMenu, 5, , acMenuVer70
  Exit_cmdRefresh_Click:
  Exit Sub
  Err_cmdRefresh_Click:
  MsgBox Err.Description
  Resume Exit_cmdRefresh_Click
End Sub

'#########################################################
'VBA code to import CPET results into database
'#########################################################
Option Compare Database
Private Sub cmdImport_Click()
  Dim strEmptyline As String
  On Error Resume Next
  Dim strResult As String
  Dim strResult2 As String
  Dim strTemp() As String
  Dim channel As Long
  channel = TermChannel
  strResult = TermGetData(channel)
  If strResult = "" Then Exit Sub
  strResult = Replace(strResult, "qq", "")
  'strTemp = Split(strResult, "COMMENT SCREEN")
  ' strResult = strTemp(1)
  'strResult = strTemp(0)
  strTemp = Split(strResult, vbCrLf)
  For j = 0 To UBound(strTemp)
  If Trim(strTemp(j)) <> "" Then
   Me.txtFull.SetFocus
   Me.txtFull.Value = Me.txtFull.Value & Trim(strTemp(j)) & vbCrLf
  'Debug.Print Trim(strTemp(j))
  End If
  Next j
  Me.Application.DDEPoke channel, "qvt_term_item", "1"
  WaitFor (1)
  DDETerminateAll
End Sub
Function TermGetData(ChannelNum As Long) As String
  On Error GoTo ErrorHandler
  Dim RLItem As String
  RLItem = "Table 0,0,80,24,Comma"
  TermGetData = DDERequest(ChannelNum, RLItem)
  Exit Function
  ErrorHandler:
  MsgBox "Error capturing data from RealLink   : " & _
  Err.Description, vbOKOnly, "Error"
End Function
Function TermChannel() As Long
  Dim AppName As String
  Dim Topic As String
  AppName = "qvt_term"
  Topic = "Respiratory"
  DDETerminateAll
  On Error GoTo ErrorHandler
  TermChannel = DDEInitiate(AppName, Topic)
  Exit Function
  ErrorHandler:
  MsgBox "Error initiating connection to Term   Error ID: " & _
  Err.Description, vbOKOnly, "Error"
End Function
Sub TermSendData(strdata As String, WaitSecs As Integer)
  On Error GoTo ErrorHandler
  Dim starttime As Double
  AppActivate "RealLink   NGTBIO", Wait
  SendKeys strdata, True
  starttime = Timer
  While Timer < starttime + WaitSecs: Wend
  Exit Sub
  ErrorHandler:
'Error gets dealt with by procedure calling this function
End Sub
Private Sub cmdSendPatient_Click()
  Dim channel As Long
  Dim strName As String
  strName = "ret" & vbCr & "7" & Me.Parent!Surname & " " _
   & Me.Parent!Forename & vbCr
  channel = TermChannel
  Me.Application.DDEPoke channel, "qvt_term_item", strName
  DDETerminateAll
End Sub
\end{lstlisting}

