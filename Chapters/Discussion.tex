% Discussion

\chapter{Discussion}
\label{ch_discussion}

\lhead{Chapter \ref{ch_discussion}. \emph{Discussion}} % This is for the header on each page - perhaps a shortened title

\clearpage
%----------------------------------------------------------------------------------------
\section{Aims of thesis}

The overall aim of the thesis was to examine the inter-relationships between preoperative clinico-pathological characteristics including cardiopulmonary exercise physiology, obstructive jaundice, body composition and preoperative systemic inflammation and post-operative complications and the post-surgical systemic inflammatory response in patients undergoing pancreaticoduodenectomy.

The present work also examined the factors that affect cardiopulmonary exercise physiology in order to enable a better understanding of the complex pathophysiology in these patients that is a consequence of the interaction of the patient's chronic conditions with the acute derangements brought on by pancreatic disease, including obstructive jaundice.

%################################################################################
\section{Clinical utility of CPET in predicting postoperative complications}
%################################################################################
The results presented in chapter \ref{ch_cpet_outcomes} demonstrated that cardiopulmonary exercise testing has a role in identifying high risk patients who are more likely to develop complications, stay longer in hospital and are less likely to receive adjuvant therapy after pancreaticoduodenectomy. 
These findings have since been replicated by other authors examining the role of cardiopulmonary exercise testing in pancreatic surgery. 

Ausania and co-workers evaluated the role of cardiopulmonary exercise testing in 122 patients who underwent a pancreaticoduodenectomy \parencite{ausania_effects_2012}
Low $\dot{V}_{O_2}$AT ($<$10.1 ml/kg/min) was the only independent predictor of a postoperative pancreatic fistula.
The incidence of a postoperative pancreatic fistula was 45\% in patients with a low $\dot{V}_{O_2}$AT while it was only 19\% in patients with a normal $\dot{V}_{O_2}$AT (p=0.02).
The association between $\dot{V}_{O_2}$AT and postoperative pancreatic fistula was independent of pancreatic duct size, body mass index, obstructive jaundice or preoperative biliary drainage.
They also reported in a separate study of 50 patients that complication  rates were higher in patients with a low $\dot{V}_{O_2}$AT undergoing palliative surgical bypass for advanced pancreatic cancer \parencite{ausania_double_2012}.

Junejo and co-workers identified elevated $\dot{V}_E/\dot{V}_{CO_2}$ to be an independent predictor of 30-day mortality after pancreaticoduodenectomy in their study that included 64 patients who had undergone cardiopulmonary exercise testing \parencite{junejo_cardiopulmonary_2014}.
$\dot{V}_{O_2}$AT and $\dot{V}_{O_2}$Peak  were not related to mortality or complications in this study.
However, they also noted that a high $\dot{V}_E/\dot{V}_{CO_2}$ was also associated with poor long-term survival.



The results of our work as well as that of several others over the past 2 decades support the use of cardiopulmonary exercise testing as a clinical risk assessment tool in patients undergoing major surgery.
However, as presented in the subsequent chapters of this thesis, a better understanding of the determinants of aerobic capacity as well as the perioperative systemic inflammatory response will enable clinicians to identify high risk patients and optimise their perioperative care.

% Different surgeries and patient groups may need different thresholds for risk prediction
% CPET and long-term survival in PDAC
% Why is low $\dot{V}_{O_2}$ likely to leak - pancreatic perfusion 
high lactate - popf de_schryver_early_2015

karoliska - ansorge_early_2012


%################################################################################
\section{Determinants of aerobic capacity}
%################################################################################
In Chapters \ref{ch_cpet_jaundice} and \ref{ch_bodycomp} we examined the preoperative patient factors that had an adverse effect on aerobic capacity.
A better understanding of the factors that affect $\dot{V}_{O_2}$ may identify potential therapeutic targets or modifiable risk factors that may be used to improve patient fitness and mitigate the risks of surgery.
We found that female sex, high body mass index, anaemia, presence of cancer and elevated CRP were all associated with either a low $\dot{V}_{O_2}$AT or a low $\dot{V}_{O_2}$Peak.

In our opinion, the most important observation made in chapter \ref{ch_cpet_jaundice} is the lack of association between obstructive jaundice and poor aerobic fitness. 
We are aware of only one other published study looking at cardiopulmonary exercise testing and peripheral oxygen extraction in jaundiced versus non-jaundiced patients undergoing pancreaticoduodenectomy. 
Junejo and co-workers reported that peripheral oxygen extraction was normal during exercise in patients with malignant obstructive jaundice.\parencite{junejo_peripheral_2014} 

Preoperative biliary drainage is unlikely to improve aerobic fitness or modify their cardiopulmonary response to exercise/surgery.
Moreover, several studies have shown that preoperative biliary drainage increases the incidence of postoperative complications, especially infective complications. \parencite{van_der_gaag_preoperative_2010, arkadopoulos_preoperative_2014, fujii_preoperative_2015, furukawa_negative_2015}
Our results therefore support the view that routine preoperative biliary drainage should no longer be performed in all patients scheduled to undergo a pancreaticoduodenectomy.

Overweight and obese patients had significantly lower $\dot{V}_{O_2}$AT and $\dot{V}_{O_2}$Peak in spite of no significant increase in known cardiac or respiratory comorbidities. 
We present a detailed analysis of the relationship between body composition and preoperative cardiopulmonary exercise testing in chapter \ref{ch_bodycomp}.
This has not been reported before in surgical patients to the best of out knowledge. 

The loss of skeletal muscle mass in patients with pancreatic cancer and spurious correlation with obesity due to correction for total body weight may partly explain the low $\dot{V}_{O_2}$ in these patients. 
Low aerobic capacity as measured by cardiopulmonary exercise testing must be interpreted with caution in overweight/obese patients and in patients with pancreatic cancer. 
This is especially true for $\dot{V}_{O_2}$ that has been scaled for total body weight.
Further studies must evaluate other parameters such as $O_2$Pulse or $\dot{V}_E/\dot{V}_{O_2}$ for their ability to predict postoperative outcomes. 
With the significant increase in the proportion of obese patients undergoing major surgery and the presence of sarcopenic-obesity in patients with pancreatic cancer, poor performance at cardiopulmonary exercise testing may not necessarily be related to cardiac or pulmonary function. 

Moreover, low aerobic capacity was not associated with documented cardiovascular or respiratory comorbidity or with an elevated POSSUM physiology score.
It is possible that most patients with significant cardiac or respiratory comorbidity were identified as unsuitable for surgery and therefore were never subject to cardiopulmonary exercise testing. 
Alternatively, it is possible that these patients did very poorly at cardiopulmonary exercise testing and this, in addition to their cardiorespiratory medical history precluded them from surgery.
We did not include any patients who did not have an operation and this may have introduced a selection bias.

%################################################################################
\section{The role of systemic inflammation}
%################################################################################
We reported on the preoperative factors affecting postoperative systemic inflammation in chapter \ref{ch_pre_post_sirs} and the clinical utility of monitoring trends in the early postoperative systemic inflammatory response in predicting complications in chapter \ref{ch_crp_comp}.

Systemic inflammation is increasingly recognised as playing an important role in determining not only long-term outcomes but also short-term outcomes after cancer surgery.
Several recent studies have demonstrated the value of monitoring trends in postoperative CRP levels in predicting complications after pancreaticoduodenectomy.
These studies performed at large, specialist centres, have used a combination of postoperative CRP and drain amylase levels to identify patients who are at the least risk of developing a Grade B/C pancreatic fistula \parencite{hiyoshi_usefulness_2013, ansorge_diagnostic_2014, kosaka_multivariate_2014}.

\bigskip \hrule

ANYTHING AFTER THIS IS IN EARLY DRAFT STAGE

\bigskip \hrule

\section{Improving outcomes after pancreaticoduodenectomy}



- centralisation
- specialisation
- early rescue and aggressive management of complications \parencite{gouma_rates_2000}

- better understanding of physiology
- role of sytemic inflammation \parencite{van_heek_hospital_2005, ho_effect_2003, birkmeyer_surgeon_2003, halm_is_2002}
- impact of sarcopenic-obesity \parencite{joglekar_sarcopenia_2015, reisinger_sarcopenia_2015, gonzalez_obesity_2014}
- role of tumour related factors \parencite{williams_surgical_2014}

\subsection{Prehabilitation}

%using cpet to assess impact of neo-adjuvant treatment
%recommend prehab for neoadj in pancreatic cancer
%different types of surgery may need different levels of preparation

Recognition of these factors is important not only in patient selection and informed consent, but also in instituting the appropriate interventions as part of prehabilitation programmes. 
In a recent study of 8266 patients undergoing pancreaticoduodenectomy in the United States, Tzeng and co-workers reported that a third (3033) were of borderline fitness as a consequence of advanced age ($>$ 80), poor performance status, weight loss $>$ 10\%, pulmonary disease, recent myocardial infarction/angina, stroke history, and/or preoperative sepsis. 
The authors report that major complications (31.3 vs. 26.2\%) and mortality (4.1 vs. 2.3\%) were greater in these patients \parencite{tzeng_morbidity_2014}.
The authors recommend that surgeons identify these patients early, institute interventions to optimise their comorbidities and enrol these patients for prehabilitation. 

West and co-workers used prehabilitation to return aerobic capacity to baseline levels in patients with locally advanced rectal cancer undergoing neo-adjuvant chemoradiotherapy \parencite{west_effect_2015}.
Moderate aerobic and resistance exercises and protein supplementation started 4 weeks before surgery was noted to improve postoperative functional exercise capacity in patients undergoing colorectal cancer surgery \parencite{gillis_prehabilitation_2014}.
Exercise programs as short as 4 weeks in duration have been reported to improve aerobic capacity measured objectively by cardiopulmonary exercise testing by upto 10\% which can potentially move approximately 30\% of high risk patients into the low risk category \parencite{dunne_pmo-029_2012}.

\subsection{Intra-operative care}

The results of chapter \ref{ch_cpet_outcomes} and the work of Ausania and co-workers appeared to suggest a link between low aerobic capacity and increased incidence of pancreatic fistula. We hypothesised that tissue hypoperfusion may have played a role in the poor healing of the pancreatico-jejunal anastomosis. 
Reyad and co-workers performed a randomised trial to examine the effect of intra-operative dobutamine infusion during pancreaticoduodenectomy on splanchnic perfusion, hemodynamics, and overall postoperative outcome \parencite{reyad_effect_2013}. 
They reported that intra-operative dobutamine use was associated with improved global oxygen delivery as measured by arterial and venous blood gases, splanchnic perfusion as measured by gastric tonometry and postoperative complications.
It was interesting to note that the incidence of anastomotic leak was 30\% in the control group, 15\% in the group that received 3$\mu$g/kg/min of dobutamine and 5\% in the group that received 5$\mu$g/kg/min of dobutamine (p$<$0.05). The overall complication rate also decreased from 70\% in the control group to 40\% and 20\% in the dobutamine groups (p$<0.05$).

However, a recent randomised controlled trial evaluated the role of goal-directed therapy in high-risk patients undergoing major elective surgery that targeted individualised oxygen delivery \parencite{ackland_individualised_2015}.
The study did show that patients whose postoperative oxygen delivery was similar to their preoperative values had fewer complications.
However, oxygen delivery was not influenced by goal-directed therapy as any beneficial effect of the intervention was lost with the autonomic nervous system changes that accompanied the increased intravenous fluids and inotropes in the treatment cohort.

The OPTIMISE randomised controlled trial also did not show any benefit of goal-directed therapy in high risk patients undergoing major gastrointestinal surgery \parencite{pearse_effect_2014}.








\subsection{Post-operative care}


\section{Future directions}


%Early recognition of patients at increased risk of postoperative complications facilitates prompt initiation of appropriate treatment and judicious allocation of critical care resources. 
%Moreover, identification of patients who are unlikely to develop a major postoperative complication allows their progress on enhanced recovery pathways with the benefits of rapid recovery, reduced morbidity and early discharge. 
%These patients are also more likely to receive adjuvant therapy without undue delay. 
%Recognition of these low risk patients helps improve outcomes, reduces hospital costs and releases valuable resources for the care of high risk patients. 
%The results reported in Chapter [ss] demonstrate the value of early post-operative C-reactive protein in identifying patients who are more likely to develop an infective complication. 
%The results show that CRP is only useful in the absence of a postoperative pancreatic fistula. 
%The combination of low drain amylase and low CRP on the third postoperative day identifies patients who do not have a pancreatic fistula and are less likely to develop an infective complication. 

