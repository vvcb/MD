% Chapter 06 - Conclusion

\chapter{Discussion}
\label{ch_discussion}

\lhead{Chapter \ref{ch_discussion}. \emph{Discussion}} % This is for the header on each page - perhaps a shortened title

\clearpage
%----------------------------------------------------------------------------------------

The overall aim of the thesis was to examine the relationships between cardiopulmonary exercise physiology, obstructive jaundice, systemic inflammation, body composition and short and long-term outcomes after major pancreatic surgery. 

A secondary aim of the thesis was to clarify the factors that affect cardiopulmonary exercise physiology and provide a better understanding of the complex pathophysiology in these patients that is a consequence of the interaction of the patient's chronic conditions with the acute derangements brought on by pancreatic disease, including obstructive jaundice.
%################################################################################
% Cardiopulmonary exercise testing
%################################################################################
The results of the present work have demonstrated that cardiopulmonary exercise testing provides an objective, reproducible method of identifying high risk patients who are more likely to develop complications, stay longer in hospital and are less likely to receive adjuvant therapy where the underlying pathology is a malignancy. 
These results have since been replicated by other authors examining the role of cardiopulmonary exercise testing in pancreatic surgery. \parencite{ausania_effects_2012, ausania_double_2012, junejo_cardiopulmonary_2014}

A low $\dot{V}_{O_2}$AT informs the patient, anaesthetist and surgeon of an increased perioperative risk. 
This allows everyone to make informed decisions on appropriate treatment strategies. 
For instance, a patient with excellent aerobic fitness may be able to proceed through surgery on an enhanced recovery pathway with early surgery and potentially early discharge from hospital. 
On the other hand, a patient with poor aerobic fitness as demonstrated by cardiopulmonary exercise testing will be better suited for a prehabilitation programme that involves measures such as nutritional supplementation, cessation of smoking, moderate tailored exercise, other cardiorespiratory optimisation measures including change in cardiac medication or bronchodilators. 
Some patients with extremely poor aerobic fitness in the presence of other known significant medical comorbidities may not be suitable for surgery. 
Cardiopulmonary exercise testing provides an objective, evidence-based measure of risk to rationalise treatment decisions in such high risk patients.
%################################################################################
% Obstructive Jaundice
%################################################################################
Equally important is the observation that preoperative obstructive jaundice did not impair aerobic capacity as measured by cardiopulmonary exercise testing. 
We are aware of only one other published study looking at cardiopulmonary exercise testing and peripheral oxygen extraction in jaundiced versus non-jaundiced patients undergoing pancreaticoduodenectomy. 
Junejo and co-workers reported that peripheral oxygen extraction was normal during exercise in patients with malignant obstructive jaundice.\parencite{junejo_peripheral_2014} 
Routine preoperative biliary drainage is no longer recommended before a pancreaticoduodenectomy. 
Several studies have shown that preoperative biliary drainage increases the incidence of postoperative complications, especially infective complications. \parencite{van_der_gaag_preoperative_2010, arkadopoulos_preoperative_2014, fujii_preoperative_2015, furukawa_negative_2015}
The results reported in this thesis in conjunction with recently published clinical studies support the view that routine preoperative biliary drainage should no longer be performed in all patients scheduled to undergo a pancreaticoduodenectomy.
%################################################################################
%Body composition
%################################################################################
We also observed that overweight and obese patients had significantly lower $\dot{V}_{O_2}$AT and $\dot{V}_{O_2}$Peak in spite of no significant increase in known cardiac or respiratory comorbidities. 
The analysis of the relationship between body composition and preoperative cardiopulmonary exercise testing has not been reported before in surgical patients. 
These results demonstrate that interpretation of cardiopulmonary exercise test parameters in patients with a high BMI must be made with caution, especially when such parameters have been corrected for the patient's total body weight. 
With the significant increase in the proportion of obese patients undergoing major surgery and the presence of sarcopenic-obesity in patients with pancreatic cancer, poor performance at cardiopulmonary exercise testing may not necessarily be related to cardiac or pulmonary function. 
The loss of skeletal mass and spurious correlations due to correction for total body weight may partly explain the low $\dot{V}_{O_2}$ in these patients. 

%################################################################################
%Prehabilitation
%################################################################################
Recognition of these factors is important not only in patient selection and informed consent, but also in instituting the appropriate interventions as part of prehabilitation programmes. 
In a recent study of 8266 patients undergoing pancreaticoduodenectomy in the United States, Tzeng and co-workers reported that a third (3033) were of borderline fitness as a consequence of advanced age ($>$ 80), poor performance status, weight loss $>$ 10\%, pulmonary disease, recent myocardial infarction/angina, stroke history, and/or preoperative sepsis. 
The authors report that major complications (31.3 vs. 26.2\%) and mortality (4.1 vs. 2.3\%) were greater in these patients.\parencite{tzeng_morbidity_2014} 
The authors recommend that surgeons identify these patients early, institute interventions to optimise their comorbidities and enrol these patients for prehabilitation. 

West and co-workers used prehabilitation to return aerobic capacity to baseline levels in patients with locally advanced rectal cancer undergoing neo-adjuvant chemoradiotherapy.\parencite{west_effect_2015} 
Moderate aerobic and resistance exercises and protein supplementation started 4 weeks before surgery was noted to improve postoperative functional exercise capacity in patients undergoing colorectal cancer surgery.\parencite{gillis_prehabilitation_2014}
Exercise programs as short as 4 weeks in duration have been reported to improve aerobic capacity measured objectively by cardiopulmonary exercise testing by upto 10\% which can potentially move approximately 30\% of high risk patients into the low risk category.\parencite{dunne_pmo-029_2012}

Early recognition of patients at increased risk of postoperative complications facilitates prompt initiation of appropriate treatment and judicious allocation of critical care resources. 
Moreover, identification of patients who are unlikely to develop a major postoperative complication allows their progress on enhanced recovery pathways with the benefits of rapid recovery, reduced morbidity and early discharge. 
These patients are also more likely to receive adjuvant therapy without undue delay. 
Recognition of these low risk patients helps improve outcomes, reduces hospital costs and releases valuable resources for the care of high risk patients. 
The results reported in Chapter [ss] demonstrate the value of early post-operative C-reactive protein in identifying patients who are more likely to develop an infective complication. 
The results show that CRP is only useful in the absence of a postoperative pancreatic fistula. 
The combination of low drain amylase and low CRP on the third postoperative day identifies patients who do not have a pancreatic fistula and are less likely to develop an infective complication. 