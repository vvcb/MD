% Chapter 06 - Postop CRP and complications

\chapter{An investigation into the relationship between postoperative systemic inflammation and complications after pancreaticoduodenectomy.}
\label{ch_survival}

\lhead{Chapter \ref{ch_survival}. \emph{Postoperative CRP and complications}} % This is for the header on each page - perhaps a shortened title

\clearpage
%----------------------------------------------------------------------------------------
\section{Introduction}
Pancreaticoduodenectomy is associated with significant morbidity in spite of advances in patient selection, perioperative care and surgical technique. Early identification of complications can help improve outcomes by better allocation of critical care resources as well as goal-directed therapy. Postoperative pancreatic fistula is one of the most dreaded complications after a pancreaticoduodenectomy and can lead to a cascade of other complications including delayed haemorrhage, infected intra-abdominal collections, delayed gastric emptying, prolonged hospitalisation and in some cases, death. The International Study Group for Pancreatic Fistula (ISGPF) have not only defined what constitutes a post-operative pancreatic fistula but have also graded the severity of this complication based on its impact on the management of the patient. However, these definitions are applied after the event and there is no clear way of predicting the severity of complications.

Postoperative CRP levels have been shown to predict infectious complications after colorectal surgery as well as oesophago-gastric surgery. It has been postulated that an unmitigated and exaggerated systemic inflammatory response in the early postoperative period is followed by a compensatory anti-inflammatory response that predisposes the patient to sepsis and impaired healing. The role of postoperative CRP in predicting the severity of complications after major pancreatic surgery has not been reported before. Stratifying patients based on the predicted severity of complications will allow identifying low risk patients who will be suitable to continue on enhanced recovery pathways and high risk patients who may require prolonged critical care, prolonged hospitalisation or further interventions.

The aim of this study was to investigate the association between postoperative systemic inflammation and severity of complications after pancreaticoduodenectomy.
 
 
\clearpage
\section{Methods}
Patients who underwent pancreaticoduodenectomy between August 2008 and July 2012 were included in the study. Data were collected from a prospectively maintained database. All complications were discussed at a weekly meeting and prospectively recorded. Postoperative pancreatic fistula and post-pancreatectomy haemorrhage were graded according to the International Study Group classifications. All other complications were graded using the Clavien-Dindo system. Postoperative mortality was defined as death within 30-days of the operation or while still in hospital after the operation. Re-intervention in the form of radiological, endoscopic or surgical procedures was recorded prospectively. Preoperative blood results as well as postoperative blood results for the first 7 days after surgery were collected from the hospital laboratory databases. 

\subsection{Statistics}
SPSS version 22 was used for all analysis. ROC analysis was used to identify the optimum thresholds of CRP for predicting complications.
\todo{Complete this section - most of this can be modified from the stats sections in other chapters}



\clearpage
\section{Results}



\clearpage
\section{Discussion}