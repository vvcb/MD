\chapter{An investigation into the relationship between preoperative clinico-pathological characteristics and post-operative systemic inflammatory response in patients undergoing pancreaticoduodenectomy.}
\label{ch_bodycomp}

\lhead{Chapter \ref{ch_bodycomp}. \emph{Factors affecting post-operative systemic inflammation}} % This is for the header on each page - perhaps a shortened title

\clearpage
%----------------------------------------------------------------------------------------

\section{Introduction}
The perioperative systemic inflammatory response has a significant role in determining short-term and long-term outcomes following potentially curative surgery for a wide variety of cancers. Systemic inflammation both before and after major surgery has been reported to be associated with significant morbidity. 

An exaggerated preoperative systemic inflammatory response syndrome is associated with increased complications after colorectal surgery \parencite{moyes_preoperative_2009, kubo_elevated_2013}, oesophagectomy \parencite{vashist_glasgow_2010} as well as liver surgery for colorectal metastases \parencite{neal_preoperative_2011}. 
The modified Glasgow Prognostic Score in particular, which uses the combination of C-reactive protein and serum albumin, has been reported to be associated with increased incidence of complications \parencite{moyes_preoperative_2009, mohri_correlation_2014, vashist_glasgow_2010}.
Preoperative CRP levels have also been reported to be associated with increased incidence of complications including infections and renal dysfunction as well as increased in-hospital mortality after cardiac surgery \parencite{lorenzo_increased_2012, mezzomo_preoperative_2011, kim_predictive_2009, biancari_preoperative_2003, boeken_increased_1998}.

Moreover, an exaggerated postoperative systemic inflammatory response in the first few days after surgery is associated with increased incidence of infective complications after a wide variety of thoraco-abdominal procedures\parencite{platt_c-reactive_2012, dutta_persistent_2011, welsch_persisting_2008} as well as other types of surgery\parencite{mcneer_early_2010, laporta_baez_c-reactive_2011}.
The magnitude of this postoperative inflammatory response has also been reported to be associated with the severity of the complications \parencite{mcsorley_postoperative_2015}. 

In patients undergoing pancreaticoduodenectomy, preoperative systemic inflammation may be affected by several factors including the presence of obstructive jaundice with or without cholangitis, preoperative biliary intervention including endoscopic retrograde cholangio-pancreatography for diagnosis or biliary drainage and in some patients acute or chronic pancreatitis either due to obstruction of the main pancreatic duct or due to other causes. 
The effect this `priming' of the immune system on postoperative outcomes is poorly understood in this cohort of patients. 

Chronic inflammation is a recognised feature of obesity which is increasingly common in patients undergoing major surgery for pancreatic and other gastro-intestinal cancers. 
The impact of obesity, especially visceral obesity, on complications after pancreaticoduodenectomy remains controversial with some authors reporting that obesity is associated with increased incidence of complications \parencite{house_preoperative_2008, ramsey_body_2011} while others reporting similar outcomes in obese and non-obese patients \parencite{khan_does_2010, tsai_impact_2010, balentine_obesity_2011}. 

More recently, levels of adipocytokines, inflammatory mediators produced exclusively in adipose tissue, have been reported to be associated with postoperative surgical site infections after colorectal \parencite{ortega-deballon_preoperative_2013, matsuda_preoperative_2009} and gastric cancer surgery \parencite{yamamoto_association_2013}.
This emphasises the importance of adipose tissue metabolism and body composition in the preoperative systemic inflammatory status of the patient undergoing surgery. 

To our knowledge, the relationship between the preoperative systemic inflammatory response and the magnitude of the postoperative systemic inflammatory response has not been examined before. 
While obstructive jaundice in itself has recently been reported to have no effect on postoperative complications, the impact of preoperative obstructive jaundice on postoperative systemic inflammation has not been reported before. 
Moreover, the relationship between comorbidity, body composition and aerobic capacity as measured by cardiopulmonary exercise testing and postoperative systemic inflammation has not been studied. 

The aim of this study was to examine the relationship between patient factors including preoperative systemic inflammation, obstructive jaundice, cardiopulmonary exercise test parameters and body composition and the magnitude of the postoperative systemic inflammation during the first week after a pancreaticoduodenectomy. 

%Post-operative CRP levels have been reported to be associated with the magnitude of surgery as well as to predict infectious complications after neurosurgery \parencite{al-jabi_value_2010}. 


\section{Methods}

\section{Results}

\begin{sidewaystable}[p]
	\caption{The relationship  between postoperative CRP and preoperative clinico-pathological characteristics in patients undergoing pancreaticoduodenectomy. }
	\label{table:sirs_crp}
	\footnotesize
	\centering
	\renewcommand{\arraystretch}{1.2} %Increases space between rows
	%\setlength{\tabcolsep}{9pt} %sets the space between columns

	\begin{tabular}{|llr | cccccccc|}
		\hline
		Preop.              &           &   n &                                   \multicolumn{8}{c|}{Postoperative C-Reactive Protein}                                   \\
		variable            &           & 188 & Day 0      &     Day 1     &     Day 2     &     Day 3     &     Day 4     &     Day 5     &    Day 6     &     Day 7     \\ \hline
		CRP                 & $\leq$10  & 116 & 18 (11-31) & 115 (86-145)  & 214 (166-262) & 181 (132-245) & 142 (85-216)  & 114 (60-193)  & 109 (56-175) & 103 (55-175)  \\
		                    & $>$10     &  70 & 39 (26-56) & 151 (99-186)  & 211 (158-282) & 195 (145-252) & 152 (89-231)  & 114 (61-197)  & 105 (55-162) & 109 (50-172)  \\
		Albumin             & $\geq$35  &  92 & 25 (13-39) & 126 (95-157)  & 220 (165-279) & 205 (151-276) & 170 (103-241) & 131 (83-227)  & 140 (71-204) & 124 (72-192)  \\
		                    & $<$35     &  96 & 26 (14-44) & 119 (87-166)  & 204 (161-253) & 175 (124-237) & 134 (83-209)  & 101 (47-155)  & 87 (44-159)  &  88 (41-151)  \\
		mGPS                & 0         & 116 & 18 (11-31) & 115 (86-145)  & 214 (166-262) & 181 (132-245) & 142 (85-216)  & 114 (60-193)  & 109 (56-175) & 103 (55-175)  \\
		                    & 1         &  16 & 34 (25-72) & 164 (130-185) & 279 (196-304) & 229 (179-309) & 194 (128-281) & 164 (112-283) & 133 (77-226) & 127 (72-226)  \\
		                    & 2         &  54 & 39 (27-53) & 145 (95-186)  & 200 (152-248) & 179 (124-241) & 140 (79-216)  & 102 (58-153)  & 96 (54-159)  & 109 (44-154)  \\
		NLR                 & $\leq$5   & 159 & 24 (13-38) & 121 (88-161)  & 218 (165-270) & 186 (136-258) & 148 (85-226)  & 115 (60-197)  & 113 (56-175) & 117 (57-176)  \\
		                    & $>$5      &  29 & 43 (19-66) & 141 (95-167)  & 189 (152-232) & 160 (124-236) & 141 (92-189)  & 113 (67-144)  & 89 (54-149)  &  89 (50-119)  \\
		Bilirubin           & $\leq$35  & 113 & 25 (13-41) & 131 (95-165)  & 231 (180-275) & 204 (154-273) & 171 (102-237) & 127 (81-218)  & 121 (72-192) & 113 (71-175)  \\
		                    & 36-250    &  44 & 24 (14-43) & 124 (100-167) & 212 (167-277) & 182 (139-239) & 136 (90-203)  & 110 (51-139)  & 113 (48-163) & 103 (44-174)  \\
		                    & $>$250    &  31 & 28 (17-37) &  99 (67-128)  & 165 (116-202) & 149 (82-214)  &  95 (60-166)  &  59 (37-153)  & 55 (26-154)  &  70 (24-158)  \\
		BMI                 & $<25$     &  80 & 21 (12-42) & 121 (88-163)  & 207 (156-252) & 174 (132-242) & 134 (86-217)  & 105 (58-193)  & 87 (51-175)  &  92 (50-164)  \\
		                    & 25-29.9   &  64 & 26 (17-39) & 127 (93-164)  & 220 (187-269) & 203 (137-250) & 149 (86-201)  & 115 (62-164)  & 109 (55-165) & 110 (54-170)  \\
		                    & 30-34.9   &  23 & 34 (12-42) & 114 (90-133)  & 220 (148-270) & 205 (152-287) & 190 (97-230)  & 141 (67-222)  & 142 (55-181) & 136 (70-198)  \\
		                    & $>$35     &   9 & 31 (27-44) & 167 (116-178) & 204 (173-311) & 235 (147-293) & 210 (91-290)  & 167 (76-232)  & 172 (88-207) & 162 (148-178) \\
		%bmi                & $\leq$25  &  80 & 21 (12-42) & 121 (88-163)  & 207 (156-252) & 174 (132-242) & 134 (86-217)  & 105 (58-193)  & 87 (51-175)  &  92 (50-164)  \\
		%                   & $>$25     &  96 & 27 (17-40) & 124 (93-162)  & 220 (170-270) & 205 (148-266) & 164 (90-217)  & 118 (66-192)  & 117 (57-172) & 122 (57-175)  \\
		SIMD                & 4-5       &  61 & 26 (13-45) & 128 (98-168)  & 226 (183-271) & 208 (150-274) & 171 (108-231) & 127 (88-200)  & 137 (72-205) & 136 (71-220)  \\
		                    & 1-3       & 126 & 25 (14-39) & 121 (88-161)  & 207 (148-264) & 180 (132-242) & 140 (84-215)  & 105 (57-190)  & 94 (50-163)  &  98 (50-158)  \\
		$\dot{V}_{O_2}$AT   & $\geq$10  &  77 & 23 (12-40) & 118 (85-165)  & 222 (166-264) & 186 (132-252) & 148 (85-231)  & 111 (58-193)  & 106 (54-177) & 107 (51-173)  \\
		                    & $<$10     &  52 & 28 (19-41) & 121 (96-150)  & 222 (158-275) & 204 (151-262) & 175 (123-221) & 119 (80-204)  & 116 (63-174) & 113 (66-189)  \\
		$\dot{V}_{O_2}$Peak & $\geq$160 &  65 & 21 (12-37) & 106 (78-150)  & 214 (140-264) & 181 (132-235) & 150 (85-213)  & 110 (59-190)  & 113 (57-181) & 115 (51-175)  \\
		                    & $<$16     &  65 & 28 (19-44) & 123 (100-161) & 234 (173-279) & 214 (149-278) & 174 (92-237)  & 119 (63-204)  & 111 (57-172) & 111 (60-180)  \\
		Hb                  & $\geq$12  & 127 & 24 (13-38) & 116 (85-161)  & 214 (165-263) & 184 (136-247) & 151 (88-214)  & 111 (59-192)  & 102 (54-172) & 104 (51-174)  \\
		                    & $<$12     &  61 & 28 (18-45) & 140 (104-165) & 216 (163-279) & 186 (124-258) & 137 (89-231)  & 119 (64-197)  & 119 (63-170) & 106 (60-170)  \\
		PPS                 & $\leq$14  &  93 & 24 (13-36) & 116 (84-161)  & 205 (144-264) & 180 (129-242) & 131 (83-210)  & 107 (55-168)  & 104 (54-174) & 104 (52-174)  \\
		                    & $>$14     &  87 & 26 (14-45) & 128 (99-156)  & 214 (175-262) & 198 (136-255) & 158 (95-226)  & 118 (71-193)  & 114 (58-167) & 109 (53-173)  \\ \hline
	\end{tabular}	
\end{sidewaystable}
































\begin{sidewaystable}[p]
	\caption{The relationship  between postoperative Albumin and preoperative clinico-pathological characteristics in patients undergoing pancreaticoduodenectomy. }
	\label{table:sirs_alb}
	\footnotesize
	\centering
	\renewcommand{\arraystretch}{1.2} %Increases space between rows
	%\setlength{\tabcolsep}{9pt} %sets the space between columns

	\begin{tabular}{|llr | c c c c c c c c|}
		\hline
		Preop.              &           &   n &                           \multicolumn{8}{c|}{Postoperative Serum Albumin}                            \\
		variable            &           & 188 & Day 0      & Day 1      & Day 2      & Day 3      & Day 4      & Day 5      & Day 6      & Day 7      \\ \hline
		CRP                 & $\leq$10  & 116 & 19 (16-23) & 20 (18-23) & 20 (18-23) & 19 (17-22) & 20 (16-22) & 20 (17-23) & 20 (17-23) & 20 (17-24) \\
		                    & $>$10     &  70 & 16 (13-19) & 17 (14-21) & 17 (15-20) & 16 (14-19) & 16 (14-19) & 17 (15-19) & 17 (15-20) & 18 (15-22) \\
		Albumin             & $\geq$35  &  92 & 20 (18-23) & 22 (19-24) & 22 (20-23) & 20 (19-23) & 20 (18-23) & 21 (18-23) & 21 (19-24) & 21 (19-25) \\
		                    & $<$35     &  96 & 16 (13-19) & 16 (14-20) & 17 (14-19) & 16 (13-19) & 16 (14-19) & 16 (14-19) & 17 (15-20) & 17 (14-20) \\
		mGPS                & 0         & 116 & 19 (16-23) & 20 (18-23) & 20 (18-23) & 19 (17-22) & 20 (16-22) & 20 (17-23) & 20 (17-23) & 20 (17-24) \\
		                    & 1         &  16 & 20 (18-22) & 22 (19-23) & 21 (19-22) & 20 (18-22) & 19 (17-21) & 18 (17-23) & 20 (18-23) & 21 (17-24) \\
		                    & 2         &  54 & 14 (13-17) & 16 (14-18) & 16 (14-18) & 15 (13-19) & 15 (14-19) & 16 (14-18) & 16 (14-20) & 17 (14-20) \\
		NLR                 & $\leq$5   & 159 & 19 (15-22) & 20 (16-23) & 20 (17-22) & 19 (16-22) & 19 (15-22) & 19 (16-22) & 19 (16-23) & 20 (16-23) \\
		                    & $>$5      &  29 & 15 (12-18) & 17 (15-19) & 17 (15-20) & 16 (14-19) & 17 (14-19) & 17 (15-19) & 18 (16-20) & 19 (15-22) \\
		Bilirubin           & $\leq$35  & 113 & 20 (17-23) & 21 (18-24) & 21 (19-23) & 20 (18-22) & 20 (18-22) & 20 (18-23) & 20 (18-23) & 21 (18-23) \\
		                    & 36-250    &  44 & 16 (14-20) & 18 (14-21) & 17 (16-20) & 16 (15-19) & 17 (14-20) & 17 (15-22) & 18 (15-23) & 19 (15-23) \\
		                    & $>$250    &  31 & 14 (12-16) & 15 (13-17) & 15 (12-17) & 14 (12-16) & 14 (13-16) & 15 (13-17) & 15 (14-17) & 15 (13-19) \\
		BMI                 & $<25$     &  80 & 17 (14-20) & 19 (15-22) & 18 (16-22) & 18 (15-20) & 18 (15-21) & 18 (15-22) & 19 (15-22) & 19 (15-23) \\
		                    & 25-29.9   &  64 & 19 (14-21) & 20 (16-23) & 20 (16-23) & 19 (15-22) & 19 (16-21) & 19 (17-22) & 20 (17-22) & 20 (17-23) \\
		                    & 30-34.9   &  23 & 19 (16-23) & 19 (16-22) & 20 (17-22) & 19 (16-21) & 19 (16-21) & 20 (17-22) & 19 (17-22) & 20 (19-23) \\
		                    & $>$35     &   9 & 20 (17-23) & 19 (18-26) & 20 (18-22) & 19 (18-20) & 19 (17-20) & 17 (16-21) & 18 (16-20) & 18 (17-20) \\
		%bmi                & $\leq$25  &  80 & 17 (14-20) & 19 (15-22) & 18 (16-22) & 18 (15-20) & 18 (15-21) & 18 (15-22) & 19 (15-22) & 19 (15-23) \\
		%                   & $>$25     &  96 & 19 (15-22) & 19 (16-23) & 20 (17-22) & 19 (16-21) & 19 (16-21) & 19 (17-22) & 19 (17-22) & 20 (17-23) \\
		SIMD                & 4-5       &  61 & 18 (16-23) & 20 (17-23) & 20 (17-23) & 19 (16-22) & 19 (16-21) & 18 (16-21) & 19 (17-22) & 19 (16-22) \\
		                    & 1-3       & 126 & 18 (14-21) & 19 (15-22) & 20 (16-22) & 19 (15-21) & 19 (15-21) & 18 (15-22) & 19 (15-23) & 20 (16-23) \\
		$\dot{V}_{O_2}$AT   & $\geq$10  &  77 & 19 (16-23) & 20 (18-23) & 20 (18-22) & 19 (16-22) & 19 (16-21) & 19 (17-22) & 20 (17-22) & 20 (16-23) \\
		                    & $<$10     &  52 & 17 (13-20) & 18 (15-22) & 17 (15-22) & 16 (14-20) & 17 (14-21) & 17 (14-20) & 17 (14-20) & 18 (15-21) \\
		$\dot{V}_{O_2}$Peak & $\geq$160 &  65 & 20 (16-23) & 21 (17-23) & 20 (18-23) & 19 (16-22) & 19 (16-22) & 19 (17-23) & 20 (16-23) & 20 (16-23) \\
		                    & $<$16     &  65 & 18 (14-20) & 19 (15-21) & 18 (15-22) & 17 (15-20) & 18 (14-20) & 18 (15-20) & 18 (15-20) & 19 (15-22) \\
		Haemoglobin         & $\geq$12  & 127 & 19 (16-22) & 20 (17-23) & 20 (18-23) & 19 (17-22) & 19 (16-22) & 20 (17-23) & 20 (17-23) & 20 (17-24) \\
		                    & $<$12     &  61 & 16 (13-19) & 17 (14-20) & 17 (14-20) & 16 (13-19) & 16 (14-19) & 17 (14-18) & 17 (15-20) & 17 (15-21) \\
		PPS                 & $\leq$14  &  93 & 19 (16-23) & 20 (17-23) & 20 (18-23) & 19 (17-22) & 20 (16-22) & 20 (17-23) & 20 (16-23) & 21 (17-25) \\
		                    & $>$14     &  87 & 17 (14-20) & 18 (15-21) & 18 (15-21) & 16 (13-20) & 17 (14-20) & 17 (14-20) & 18 (15-22) & 19 (15-22) \\ \hline
	\end{tabular}	
\end{sidewaystable}








































\begin{table}[p]
	\caption{The relationship  between postoperative C-reactive protein and preoperative clinicopathological characteristics in patients undergoing pancreaticoduodenectomy: p-values only. }
	\label{table:sirs_crp_pvalues}
	\footnotesize
	\centering
	\renewcommand{\arraystretch}{1.2} %Increases space between rows
	%\setlength{\tabcolsep}{9pt} %sets the space between columns

	\begin{tabular}{|l | c c c c c c c c|}
		\hline
		Preop.              &         \multicolumn{8}{c|}{Postoperative C-Reactive Protein}          \\
		Variable            & Day 0    & Day 1    & Day 2 & Day 3 & Day 4 & Day 5    & Day 6 & Day 7 \\ \hline
		CRP                 & $<$0.001 & $<$0.001 & 0.669 & 0.522 & 0.741 & 0.831    & 0.789 & 0.834 \\
		Albumin             & 0.445    & 0.916    & 0.148 & 0.045 & 0.018 & 0.001    & 0.002 & 0.006 \\
		mGPS                & $<$0.001 & 0.001    & 0.037 & 0.048 & 0.084 & 0.029    & 0.215 & 0.347 \\
		NLR                 & 0.001    & 0.310    & 0.143 & 0.217 & 0.490 & 0.427    & 0.227 & 0.111 \\
		Bilirubin           & 0.869    & 0.009    & 0.001 & 0.001 & 0.003 & $<$0.001 & 0.005 & 0.072 \\
		BMI                 & 0.181    & 0.312    & 0.744 & 0.376 & 0.424 & 0.504    & 0.556 & 0.214 \\
		%BMI    01          & 0.057    & 0.910    & 0.325 & 0.135 & 0.474 & 0.375    & 0.426 & 0.166 \\
		SIMD                & 0.962    & 0.399    & 0.277 & 0.243 & 0.163 & 0.422    & 0.849 & 0.713 \\
		$\dot{V}_{O_2}$AT   & 0.042    & 0.749    & 0.838 & 0.587 & 0.330 & 0.448    & 0.659 & 0.389 \\
		$\dot{V}_{O_2}$Peak & 0.022    & 0.050    & 0.122 & 0.154 & 0.218 & 0.537    & 0.992 & 0.527 \\
		Haemoglobin         & 0.025    & 0.078    & 0.735 & 0.973 & 0.905 & 0.838    & 0.682 & 0.987 \\
		PPS                 & 0.114    & 0.192    & 0.525 & 0.308 & 0.127 & 0.338    & 0.954 & 0.919 \\ \hline
		\multicolumn{9}{l}{\textit{p} - Mann-Whitney U test or Kruskal-Wallis test}
	\end{tabular}	
\end{table}
\begin{sidewaystable}[p]
	\caption{The relationship  between postoperative white cell count and preoperative clinico-pathological characteristics in patients undergoing pancreaticoduodenectomy }
	\label{table:sirs_wcc}
	\footnotesize
	\centering
	\renewcommand{\arraystretch}{1.2} %Increases space between rows
	\setlength{\tabcolsep}{5pt} %sets the space between columns
	
	\begin{tabular}{|l l | cc cc cc cc|}
		\hline
		Preop.              & n         &                                            \multicolumn{8}{c|}{Postoperative White Cell Count}                                             \\
		Variable            & 188       & Day 0           &      Day 1      & Day 2           &     Day 3      & Day 4          &     Day 5      & Day 6           &      Day 7      \\ \hline
		CRP                 & $\leq$10  & 11.7(9.3-14.8)  & 12.3(10.7-15.6) & 13.4(10.8-16.7) & 9.9(7.9-13.3)  & 8.9(6.4-11.4)  & 9.1(6.5-12.5)  & 10.6(8.0-14.7)  & 12.5(9.3-16.0)  \\
		                    & $>$10     & 12.5(10.2-15.6) & 14.1(11.1-17.7) & 14.9(11.9-18.3) & 12.8(9.5-15.3) & 10.3(8.0-14.1) & 11.0(8.8-13.5) & 13.4(9.9-16.6)  & 14.6(11.6-18.7) \\
		Albumin             & $\geq$35  & 12.2(9.8-14.9)  & 13.1(10.7-16.3) & 13.5(11.0-17.5) & 10.9(8.2-14.2) & 9.4(6.8-11.6)  & 9.4(7.0-12.5)  & 11.1(8.0-15.1)  & 13.3(10.0-16.4) \\
		                    & $<$35     & 12.1(9.7-15.4)  & 13.1(10.9-16.1) & 14.2(11.8-17.4) & 11.1(8.4-14.7) & 9.6(7.2-12.9)  & 10.2(8.1-13.2) & 12.6(9.4-16.0)  & 13.8(10.4-18.1) \\
		mGPS                & 0         & 11.7(9.3-14.8)  & 12.3(10.7-15.6) & 13.4(10.8-16.7) & 9.9(7.9-13.3)  & 8.9(6.4-11.4)  & 9.1(6.5-12.5)  & 10.6(8.0-14.7)  & 12.5(9.3-16.0)  \\
		                    & 1         & 12.3(10.4-15.3) & 15.1(10.4-17.4) & 15.7(11.6-18.5) & 13.0(8.8-15.4) & 10.0(6.8-13.0) & 10.1(8.2-11.7) & 12.3(9.4-15.0)  & 13.1(11.1-17.0) \\
		                    & 2         & 12.5(10.2-15.8) & 13.7(11.2-17.8) & 14.9(11.9-18.2) & 12.7(9.6-15.3) & 10.8(8.1-15.0) & 11.0(8.9-14.0) & 14.5(10.0-16.6) & 14.8(11.7-19.7) \\
		NLR                 & $\leq$5   & 12.2(9.7-14.9)  & 13.3(10.9-16.4) & 13.7(11.0-17.4) & 10.9(8.3-14.2) & 9.5(6.9-12.3)  & 9.8(7.4-13.0)  & 12.0(8.6-15.9)  & 13.6(10.1-17.6) \\
		                    & $>$5      & 11.8(10.4-15.6) & 11.9(9.5-15.1)  & 14.2(12.4-18.1) & 11.5(8.6-14.3) & 8.8(6.8-14.0)  & 9.2(8.0-12.3)  & 11.0(9.5-15.2)  & 13.2(10.5-15.3) \\
		Bilirubin           & $\leq$35  & 12.2(9.2-15.7)  & 13.2(10.6-16.3) & 13.7(10.8-17.4) & 10.9(8.2-14.0) & 9.5(6.9-12.2)  & 9.5(7.2-12.4)  & 11.0(8.6-15.2)  & 13.2(10.1-16.0) \\
		                    & 36-250    & 12.2(9.9-14.5)  & 13.7(10.9-17.5) & 13.8(11.4-17.7) & 10.4(8.4-15.1) & 8.8(6.8-12.9)  & 9.6(8.1-12.7)  & 12.2(8.9-15.9)  & 13.2(9.5-17.2)  \\
		                    & $>$250    & 12.1(10.6-14.4) & 12.9(11.1-15.1) & 14.2(11.9-17.0) & 11.8(8.3-13.8) & 10.4(7.2-12.7) & 12.3(7.8-16.5) & 15.6(11.9-17.1) & 14.7(11.9-21.0) \\
		BMI                 & $<25$     & 11.6(9.5-14.6)  & 12.5(10.5-15.3) & 13.4(10.4-17.0) & 10.2(8.3-13.8) & 9.4(7.1-12.2)  & 9.8(7.3-12.8)  & 11.5(9.1-15.6)  & 12.7(9.8-17.3)  \\
		                    & 25-29.9   & 12.8(9.6-15.7)  & 13.5(10.7-17.6) & 15.5(12.4-18.4) & 12.1(8.7-14.9) & 10.3(6.9-12.9) & 10.4(8.2-12.9) & 11.9(9.8-15.7)  & 13.6(11.0-16.4) \\
		                    & 30-34.9   & 11.5(8.7-14.5)  & 13.0(10.9-16.1) & 13.5(10.5-18.1) & 10.4(7.6-14.3) & 8.5(7.0-11.6)  &  8.1(6.5-9.8)  & 9.5(7.3-14.7)   & 12.2(8.3-16.5)  \\
		                    & $>$35     & 13.0(10.5-16.3) & 12.2(11.4-14.5) & 13.3(12.9-14.4) & 12.7(5.9-13.0) & 7.9(6.0-12.4)  & 8.8(6.9-12.8)  & 10.3(8.7-16.2)  & 15.6(11.7-20.6) \\
		%bmi                & $\leq$25  & 11.6(9.5-14.6)  & 12.5(10.5-15.3) & 13.4(10.4-17.0) & 10.2(8.3-13.8) & 9.4(7.1-12.2)  & 9.8(7.3-12.8)  & 11.5(9.1-15.6)  & 12.7(9.8-17.3)  \\
		%                   & $>$25     & 12.3(9.8-15.7)  & 13.3(10.9-17.0) & 14.6(12.2-18.2) & 11.8(8.3-14.6) & 9.5(6.9-12.7)  & 9.6(7.6-12.8)  & 11.5(8.8-15.7)  & 13.6(10.7-16.5) \\
		SIMD                & 1-3       & 12.4(10.3-15.7) & 13.0(11.1-16.3) & 13.8(11.7-18.1) & 10.4(8.2-14.3) & 9.4(6.9-11.9)  & 9.4(7.4-13.0)  & 12.0(9.2-15.9)  & 13.6(10.6-17.9) \\
		                    & 4-5       & 11.9(9.4-14.8)  & 13.2(10.2-16.2) & 13.7(10.8-17.4) & 11.2(8.3-14.2) & 9.7(6.9-13.0)  & 9.8(7.5-12.9)  & 11.5(8.7-15.7)  & 13.5(9.8-16.9)  \\
		$\dot{V}_{O_2}$AT   & $\geq$10  & 12.6(10.2-15.6) & 13.2(10.9-17.1) & 13.8(11.1-17.4) & 11.4(8.6-14.6) & 10.2(7.7-12.2) & 9.8(8.0-13.0)  & 12.6(9.7-15.9)  & 14.2(10.6-17.1) \\
		                    & $<$10     & 11.5(9.2-14.4)  & 13.4(10.1-16.3) & 14.8(10.8-17.7) & 11.4(8.3-14.6) & 9.5(7.3-12.8)  & 9.6(7.4-13.2)  & 11.6(8.5-15.9)  & 13.6(10.5-15.9) \\
		$\dot{V}_{O_2}$Peak & $\geq$160 & 12.4(9.9-14.9)  & 12.7(10.3-16.3) & 13.8(11.1-17.4) & 10.8(8.1-14.6) & 9.7(7.7-11.5)  & 9.7(7.5-12.7)  & 12.4(9.3-16.0)  & 14.0(9.8-17.6)  \\
		                    & $<$16     & 11.6(9.7-15.3)  & 13.9(10.9-16.7) & 14.7(11.0-18.0) & 11.6(9.0-14.3) & 9.9(7.6-13.0)  & 9.8(7.5-13.3)  & 12.0(8.7-15.5)  & 13.8(10.6-16.0) \\
		Hb                  & $\geq$12  & 12.1(9.4-15.6)  & 13.2(10.6-16.3) & 13.7(10.8-17.2) & 10.5(8.3-13.5) & 9.4(6.9-11.7)  & 9.6(7.5-12.7)  & 11.1(8.7-15.7)  & 13.5(10.1-17.7) \\
		                    & $<$12     & 12.1(9.9-14.5)  & 13.0(11.5-16.3) & 14.1(11.9-18.1) & 11.7(8.1-16.0) & 9.5(7.0-14.1)  & 10.5(7.4-13.2) & 12.6(9.5-15.9)  & 13.4(10.5-17.1) \\
		PPS                 & $\leq$14  & 12.2(9.9-14.7)  & 12.6(10.9-15.7) & 13.3(10.8-16.7) & 10.2(8.1-13.4) & 9.0(6.8-11.6)  & 9.6(7.2-12.6)  & 10.9(8.2-15.3)  & 13.3(9.5-17.7)  \\
		                    & $>$14     & 11.8(9.2-15.5)  & 13.0(9.9-17.2)  & 14.1(11.8-18.1) & 11.7(8.3-15.3) & 9.5(7.0-13.6)  & 10.2(7.5-13.2) & 12.4(9.4-15.9)  & 13.5(10.4-16.0) \\ \hline
	\end{tabular}	
\end{sidewaystable}

  





\begin{sidewaystable}[p]
	\caption{The relationship  between postoperative neutrophil count and preoperative clinico-pathological characteristics in patients undergoing pancreaticoduodenectomy. }
	\label{table:sirs_neut}
	\footnotesize
	\centering
	\renewcommand{\arraystretch}{1.2} %Increases space between rows
	\setlength{\tabcolsep}{5pt} %sets the space between columns
	
	\begin{tabular}{|l l | cc cc cc cc |}
		\hline
		Preop.              & n         &                                             \multicolumn{8}{c|}{Postoperative Neutrophil Count}                                              \\
		Variable            & 188       & Day 0           &      Day 1      & Day 2            &      Day 3      & Day 4          &     Day 5      & Day 6           &      Day 7      \\ \hline
		CRP                 & $\leq$10  & 10.3 (7.8-13.1) & 10.4 (8.5-12.6) & 11.0 (8.9-13.9)  & 8.1 (6.3-10.7)  & 6.7 (4.8-9.2)  & 6.6 (4.9-9.5)  & 8.0 (6.1-10.8)  & 9.7 (7.4-13.3)  \\
		                    & $>$10     & 11.0 (9.1-14.0) & 11.2 (9.4-14.9) & 12.6 (9.1-15.8)  & 10.5 (7.1-13.3) & 7.8 (6.0-11.4) & 8.3 (6.6-11.1) & 10.4 (7.6-13.4) & 11.6 (8.4-15.2) \\
		Albumin             & $\geq$35  & 10.7 (8.3-13.5) & 10.7 (8.5-13.8) & 11.4 (8.9-15.1)  & 8.4 (6.6-12.1)  & 7.0 (5.1-9.6)  & 7.0 (5.1-9.8)  & 8.8 (6.2-11.6)  & 10.4 (7.1-13.4) \\
		                    & $<$35     & 10.4 (8.4-13.2) & 10.7 (8.7-13.7) & 12.0 (9.0-14.6)  & 8.7 (6.4-12.3)  & 7.6 (5.3-10.8) & 7.9 (5.6-10.9) & 9.4 (6.9-13.1)  & 10.5 (7.9-14.3) \\
		mGPS                & 0         & 10.3 (7.8-13.1) & 10.4 (8.5-12.6) & 11.0 (8.9-13.9)  & 8.1 (6.3-10.7)  & 6.7 (4.8-9.2)  & 6.6 (4.9-9.5)  & 8.0 (6.1-10.8)  & 9.7 (7.4-13.3)  \\
		                    & 1         & 10.0 (8.4-13.6) & 12.5 (7.8-15.2) & 13.2 (9.3-15.9)  & 10.6 (7.5-13.3) & 7.6 (5.2-10.5) & 7.7 (6.1-9.1)  & 9.6 (7.2-11.7)  & 10.6 (8.4-14.3) \\
		                    & 2         & 11.1 (9.1-14.2) & 11.2 (9.9-14.9) & 12.6 (9.1-15.7)  & 10.4 (7.1-13.3) & 8.1 (6.0-12.3) & 8.3 (6.8-11.2) & 10.8 (7.9-13.6) & 11.6 (8.4-15.7) \\
		NLR                 & $\leq$5   & 10.6 (8.3-13.5) & 10.9 (8.7-13.8) & 11.5 (8.9-14.6)  & 8.5 (6.4-12.0)  & 7.3 (5.2-10.0) & 7.4 (5.1-10.0) & 9.1 (6.2-12.4)  & 10.4 (7.7-13.7) \\
		                    & $>$5      & 10.5 (8.3-13.3) & 10.2 (7.9-13.2) & 12.6 (10.4-16.3) & 9.8 (7.1-12.3)  & 6.9 (5.2-11.4) & 7.1 (6.1-10.3) & 9.0 (7.2-12.9)  & 10.8 (8.3-13.6) \\
		Bilirubin           & $\leq$35  & 10.6 (8.3-13.7) & 10.7 (8.5-13.8) & 11.5 (8.9-14.8)  & 8.5 (6.7-12.0)  & 7.3 (5.2-9.6)  & 7.2 (5.1-9.5)  & 8.5 (6.4-11.3)  & 10.1 (7.7-13.4) \\
		                    & 36-250    & 10.4 (8.6-12.9) & 11.4 (8.6-14.5) & 11.2 (8.8-14.7)  & 7.9 (6.3-13.0)  & 6.5 (5.1-10.7) & 7.3 (5.6-9.7)  & 9.5 (6.3-13.1)  & 10.4 (7.2-13.8) \\
		                    & $>$250    & 10.8 (9.3-13.2) & 10.7 (9.5-13.2) & 12.1 (9.1-14.9)  & 9.6 (6.6-11.8)  & 7.8 (5.2-10.6) & 9.8 (5.6-13.9) & 11.3 (8.6-13.9) & 12.2 (8.9-18.5) \\
		BMI                 & $<25$     & 10.3 (8.4-12.8) & 10.6 (8.3-13.2) & 10.9 (8.4-14.1)  & 8.2 (6.4-11.8)  & 7.3 (5.0-10.1) & 7.4 (5.0-10.0) & 9.3 (6.9-12.3)  & 10.0 (7.3-13.7) \\
		                    & 25-29.9   & 11.1 (8.1-13.6) & 11.1 (8.5-14.9) & 12.4 (10.4-15.6) & 10.2 (7.1-12.5) & 7.6 (5.2-10.6) & 7.8 (6.1-10.0) & 9.0 (7.1-12.2)  & 10.5 (8.2-13.4) \\
		                    & 30-34.9   & 10.1 (7.6-13.0) & 10.3 (8.7-14.0) & 10.9 (8.7-15.7)  & 8.2 (6.2-12.1)  & 6.7 (5.5-9.0)  & 6.2 (5.0-7.9)  & 7.2 (5.7-11.0)  & 9.9 (6.3-13.8)  \\
		                    & $>$35     & 10.4 (9.7-14.0) & 10.7 (9.4-12.2) & 10.8 (10.0-12.1) & 10.3 (4.6-10.6) & 6.2 (4.4-9.7)  & 6.0 (5.1-11.4) & 8.0 (6.8-12.8)  & 12.8 (9.4-16.4) \\
		%bmi                & $\leq$25  & 10.3 (8.4-12.8) & 10.6 (8.3-13.2) & 10.9 (8.4-14.1)  & 8.2 (6.4-11.8)  & 7.3 (5.0-10.1) & 7.4 (5.0-10.0) & 9.3 (6.9-12.3)  & 10.0 (7.3-13.7) \\
		%                   & $>$25     & 10.7 (8.3-13.6) & 10.9 (8.7-14.6) & 12.0 (10.1-15.5) & 9.5 (6.7-12.3)  & 7.0 (5.3-10.5) & 7.4 (5.5-10.0) & 8.8 (6.7-12.0)  & 10.5 (8.1-13.4) \\
		SIMD                & 4-5       & 10.4 (8.3-12.9) & 11.3 (9.1-13.9) & 12 (9.4-15.7)    &  9 (6.9-13.4)   & 8 (6-11)       &  7.6 (5.5-11)  & 9 (7.1-13)      &  11.2 (8-13.7)  \\
		                    & 1-3       & 10.7 (8.4-13.9) & 10.5 (8.5-13.5) & 11.5 (8.8-14.5)  & 8.4 (6.4-11.6)  & 6.7 (5-9.6)    &  7.1 (5.1-10)  & 9 (6.2-12.2)    & 10.4 (7.7-13.7) \\
		$\dot{V}_{O_2}$AT   & $\geq$10  & 11.0 (8.8-13.7) & 11.2 (8.6-14.8) & 11.9 (9.4-14.8)  & 9.0 (6.7-12.3)  & 7.9 (5.7-9.9)  & 8.0 (5.7-10.3) & 9.4 (7.0-12.2)  & 11.2 (7.9-13.5) \\
		                    & $<$10     & 9.9 (7.6-12.7)  & 10.7 (8.0-13.7) & 12.1 (8.8-15.0)  & 9.1 (6.8-12.5)  & 7.2 (5.5-10.6) & 7.1 (5.3-10.0) & 8.9 (6.2-12.9)  & 10.4 (8.2-13.6) \\
		$\dot{V}_{O_2}$Peak & $\geq$160 & 10.9 (8.7-13.2) & 10.3 (8.5-13.8) & 11.9 (9.4-14.8)  & 8.9 (6.3-12.3)  & 7.8 (5.7-9.5)  & 7.9 (5.6-10.0) & 9.3 (6.9-12.3)  & 11.0 (7.7-14.1) \\
		                    & $<$16     & 10.0 (8.4-13.6) & 11.8 (8.5-14.7) & 12.1 (8.9-15.1)  & 9.4 (7.1-12.5)  & 7.8 (5.5-10.8) & 7.4 (5.3-10.3) & 9.3 (6.2-12.0)  & 10.4 (8.2-13.4) \\
		Hb                  & $\geq$12  & 10.5 (8.3-13.5) & 10.7 (8.4-13.8) & 11.3 (8.9-14.5)  & 8.4 (6.7-11.3)  & 7.2 (5.2-9.6)  & 7.3 (5.5-10.0) & 8.5 (6.4-12.0)  & 10.4 (7.7-13.7) \\
		                    & $<$12     & 10.8 (8.9-13.2) & 10.6 (9.5-13.9) & 12.4 (9.4-15.1)  & 9.6 (6.3-13.3)  & 7.0 (5.1-11.9) & 7.6 (5.2-10.8) & 9.8 (7.0-13.0)  & 10.9 (8.2-13.6) \\
		PPS                 & $\leq$14  & 10.5 (8.5-12.7) & 10.7 (8.7-13.0) & 10.9 (8.7-13.9)  & 8.1 (6.3-11.3)  & 6.8 (4.8-9.4)  & 7.2 (5.1-9.5)  & 8.4 (5.9-11.4)  & 10.1 (7.2-13.7) \\
		                    & $>$14     & 10.3 (7.5-13.7) & 10.5 (8.2-14.0) & 12.0 (9.4-15.7)  & 8.7 (6.6-12.8)  & 7.3 (5.3-10.8) & 7.6 (5.5-10.4) & 9.6 (7.1-13.1)  & 10.4 (8.2-13.6) \\ \hline
	\end{tabular}	
\end{sidewaystable}


























\begin{table}[p]
	\caption{The relationship  between postoperative white cell count and preoperative clinicopathological characteristics in patients undergoing pancreaticoduodenectomy: p-values only. }
	\label{table:sirs_wcc_pvalues}
	\footnotesize
	\centering
	\renewcommand{\arraystretch}{1.2} %Increases space between rows
	%\setlength{\tabcolsep}{9pt} %sets the space between columns

	\begin{tabular}{|l | c c c c c c c c|}
		\hline
		Preop.              &      \multicolumn{8}{c|}{Postoperative White Cell Count}      \\
		Variable            & Day 0 & Day 1 & Day 2 & Day 3 & Day 4 & Day 5 & Day 6 & Day 7 \\ \hline
		CRP                 & 0.181 & 0.079 & 0.055 & 0.001 & 0.004 & 0.001 & 0.001 & 0.015 \\
		Albumin             & 0.832 & 0.742 & 0.499 & 0.476 & 0.218 & 0.101 & 0.070 & 0.375 \\
		mGPS                & 0.402 & 0.211 & 0.158 & 0.005 & 0.007 & 0.002 & 0.002 & 0.024 \\
		NLR                 & 0.784 & 0.090 & 0.417 & 0.709 & 0.948 & 0.828 & 0.781 & 0.780 \\
		Bilirubin           & 0.970 & 0.700 & 0.832 & 0.721 & 0.758 & 0.058 & 0.011 & 0.048 \\
		BMI                 & 0.364 & 0.502 & 0.089 & 0.330 & 0.742 & 0.232 & 0.397 & 0.561 \\
		%BMI    01          & 0.146 & 0.244 & 0.031 & 0.223 & 0.539 & 0.732 & 0.905 & 0.543 \\
		SIMD                & 0.359 & 0.690 & 0.652 & 0.646 & 0.569 & 0.751 & 0.866 & 0.471 \\
		$\dot{V}_{O_2}$AT   & 0.094 & 0.631 & 0.861 & 0.923 & 0.988 & 0.606 & 0.515 & 0.508 \\
		$\dot{V}_{O_2}$Peak & 0.478 & 0.483 & 0.701 & 0.285 & 0.483 & 0.836 & 0.868 & 0.707 \\
		Haemoglobin         & 0.800 & 0.551 & 0.225 & 0.182 & 0.488 & 0.619 & 0.293 & 0.890 \\
		PPS                 & 0.570 & 0.807 & 0.229 & 0.143 & 0.267 & 0.362 & 0.141 & 0.981 \\ \hline
		\multicolumn{9}{l}{\textit{p} - Mann-Whitney U test or Kruskal-Wallis test}
	\end{tabular}	
	\vspace{1cm}

	
	\caption{The relationship  between postoperative neutrophil count and preoperative clinicopathological characteristics in patients undergoing pancreaticoduodenectomy: p-values only. }
	\label{table:sirs_neut_pvalues}
		\begin{tabular}{|l | c c c c c c c c|}
			\hline
			Preop.              &       \multicolumn{8}{c|}{Postoperative Neutrophil  Count}       \\
			Variable            & Day 0 & Day 1 & Day 2 & Day 3 & Day 4 & Day 5    & Day 6 & Day 7 \\ \hline
			CRP                 & 0.167 & 0.082 & 0.079 & 0.003 & 0.008 & $<$0.001 & 0.001 & 0.019 \\
			Albumin             & 0.911 & 0.901 & 0.586 & 0.572 & 0.280 & 0.097    & 0.070 & 0.302 \\
			mGPS                & 0.304 & 0.217 & 0.213 & 0.011 & 0.018 & 0.001    & 0.002 & 0.043 \\
			NLR                 & 0.832 & 0.259 & 0.253 & 0.366 & 0.699 & 0.697    & 0.729 & 0.661 \\
			Bilirubin           & 0.904 & 0.903 & 0.801 & 0.819 & 0.669 & 0.053    & 0.012 & 0.034 \\
			BMI                 & 0.614 & 0.762 & 0.134 & 0.469 & 0.904 & 0.382    & 0.602 & 0.826 \\
			%BMI    01          & 0.342 & 0.403 & 0.051 & 0.292 & 0.612 & 0.714    & 0.789 & 0.787 \\
			SIMD                & 0.305 & 0.555 & 0.596 & 0.507 & 0.541 & 0.328    & 0.877 & 0.711 \\
			$\dot{V}_{O_2}$AT   & 0.104 & 0.571 & 0.981 & 0.942 & 0.940 & 0.388    & 0.400 & 0.676 \\
			$\dot{V}_{O_2}$Peak & 0.441 & 0.481 & 0.689 & 0.335 & 0.529 & 0.720    & 0.664 & 0.711 \\
			Haemoglobin         & 0.682 & 0.590 & 0.230 & 0.225 & 0.462 & 0.639    & 0.122 & 0.969 \\
			PPS                 & 0.730 & 0.699 & 0.149 & 0.109 & 0.255 & 0.201    & 0.033 & 0.588 \\ \hline
			\multicolumn{9}{l}{\textit{p} - Mann-Whitney U test or Kruskal-Wallis test}
		\end{tabular}
\end{table}

%========================CRP vs Post-op SIRS============================================
\clearpage
\begin{figure}[p]
	\caption{Relationship between preoperative CRP levels and postoperative inflammatory markers in the first week after pancreaticoduodenectomy.}
	\label{fig:sirs_crp}
	\centering
	\begin{subfigure}{0.48\textwidth}
		\centering
		\includegraphics[width=\textwidth]{Figures/sirs_crp_crp}
		\caption{Preop. CRP vs. postop. CRP}
		\label{fig:sirs_crp_crp}
	\end{subfigure}
	\hfill
	\begin{subfigure}{0.48\textwidth}
		\centering
		\includegraphics[width=\textwidth]{Figures/sirs_crp_alb}
		\caption{Preop. CRP vs. postop. Albumin}
		\label{fig:sirs_crp_alb}
	\end{subfigure}
	
	\begin{subfigure}{0.48\textwidth}
		\centering
		\includegraphics[width=\textwidth]{Figures/sirs_crp_wcc}
		\caption{Preop. CRP vs. postop. White Cell Count}
		\label{fig:sirs_crp_wcc}
	\end{subfigure}
	\hfill
	\begin{subfigure}{0.48\textwidth}
		\centering
		\includegraphics[width=\textwidth]{Figures/sirs_crp_neut}
		\caption{Preop. CRP vs. postop. Neutrophil Count}
		\label{fig:sirs_crp_neut}
	\end{subfigure}	
\end{figure}
%==============================================================================

%========================Albumin vs Post-op SIRS============================================
\clearpage
\begin{figure}[p]
	\caption{Relationship between preoperative Albumin levels and postoperative inflammatory markers in the first week after pancreaticoduodenectomy.}
	\label{fig:sirs_alb}
	\centering
	\begin{subfigure}{0.48\textwidth}
		\centering
		\includegraphics[width=\textwidth]{Figures/sirs_alb_crp}
		\caption{Preop. Albumin vs. postop. CRP}
		\label{fig:sirs_alb_crp}
	\end{subfigure}
	\hfill
	\begin{subfigure}{0.48\textwidth}
		\centering
		\includegraphics[width=\textwidth]{Figures/sirs_alb_alb}
		\caption{Preop. Albumin vs. postop. Albumin}
		\label{fig:sirs_alb_alb}
	\end{subfigure}
	
	\begin{subfigure}{0.48\textwidth}
		\centering
		\includegraphics[width=\textwidth]{Figures/sirs_alb_wcc}
		\caption{Preop. Albumin vs. postop. White Cell Count}
		\label{fig:sirs_alb_wcc}
	\end{subfigure}
	\hfill
	\begin{subfigure}{0.48\textwidth}
		\centering
		\includegraphics[width=\textwidth]{Figures/sirs_alb_neut}
		\caption{Preop. Albumin vs. postop. Neutrophil Count}
		\label{fig:sirs_alb_neut}
	\end{subfigure}	
\end{figure}
%==============================================================================

%========================Bilirubin vs Post-op SIRS============================================
\clearpage
\begin{figure}[p]
	\caption{Relationship between preoperative obstructive jaundice and postoperative inflammatory markers in the first week after pancreaticoduodenectomy.}
	\label{fig:sirs_bilirubin}
	\centering
	\begin{subfigure}{0.48\textwidth}
		\centering
		\includegraphics[width=\textwidth]{Figures/sirs_bil_crp}
		\caption{Preop. Bilirubin vs. postop. CRP}
		\label{fig:sirs_bil_crp}
	\end{subfigure}
	\hfill
	\begin{subfigure}{0.48\textwidth}
		\centering
		\includegraphics[width=\textwidth]{Figures/sirs_bil_alb}
		\caption{Preop. Bilirubin vs. postop. Albumin}
		\label{fig:sirs_bil_alb}
	\end{subfigure}
	
	\begin{subfigure}{0.48\textwidth}
		\centering
		\includegraphics[width=\textwidth]{Figures/sirs_bil_wcc}
		\caption{Preop. Bilirubin vs. postop. White Cell Count}
		\label{fig:sirs_bil_wcc}
	\end{subfigure}
	\hfill
	\begin{subfigure}{0.48\textwidth}
		\centering
		\includegraphics[width=\textwidth]{Figures/sirs_bil_neut}
		\caption{Preop. Bilirubin vs. postop. Neutrophil Count}
		\label{fig:sirs_bil_neut}
	\end{subfigure}	
\end{figure}
%==============================================================================
%\input{Tables/sirs_crp_bodycomp}

%========================Bilirubin vs Post-op SIRS============================================
\clearpage
\begin{figure}[p]
	\caption{Relationship between preoperative $\dot{V}_{O_2}$AT and postoperative inflammatory markers in the first week after pancreaticoduodenectomy.}
	\label{fig:sirs_at}
	\centering
	\begin{subfigure}{0.48\textwidth}
		\centering
		\includegraphics[width=\textwidth]{Figures/sirs_at_crp}
		\caption{Preop. $\dot{V}_{O_2}$AT vs. postop. CRP}
		\label{fig:sirs_at_crp}
	\end{subfigure}
	\hfill
	\begin{subfigure}{0.48\textwidth}
		\centering
		\includegraphics[width=\textwidth]{Figures/sirs_at_alb}
		\caption{Preop. $\dot{V}_{O_2}$AT vs. postop. Albumin}
		\label{fig:sirs_at_alb}
	\end{subfigure}
	
	\begin{subfigure}{0.48\textwidth}
		\centering
		\includegraphics[width=\textwidth]{Figures/sirs_at_wcc}
		\caption{Preop. $\dot{V}_{O_2}$AT vs. postop. White Cell Count}
		\label{fig:sirs_at_wcc}
	\end{subfigure}
	\hfill
	\begin{subfigure}{0.48\textwidth}
		\centering
		\includegraphics[width=\textwidth]{Figures/sirs_at_neut}
		\caption{Preop. $\dot{V}_{O_2}$AT vs. postop. Neutrophil Count}
		\label{fig:sirs_at_neut}
	\end{subfigure}	
\end{figure}
%==============================================================================


\section{Discussion}