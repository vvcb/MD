\documentclass[12pt,a4paper]{article}
\usepackage[utf8]{inputenc}
\usepackage{amsmath}
\usepackage{amsfonts}
\usepackage{amssymb}
\usepackage{graphicx}
\author{Vishnu V Chandrabalan}

\renewcommand{\familydefault}{\sfdefault}

\begin{document}

\section{Chapter 3}

\subsection{OJ vs $\dot{V}_{O_2}$}
	- OJ was thought to be associated with adverse haemodynamic events\\
	- Routine PBD was advised based on this\\
	- Multiple studies have reported adverse events after PBD\\
	- Management of Bil>250 is unclear\\
	- Here, OJ not associated with VO2\\
	- Parker et al, Junejo et al\\
	- \textbf{NO JUSTIFICATION FOR PBD}\\
	-Figures on P87, Tables p91, 92

\subsection{$\dot{V}_{O_2}$ vs Female/BMI}
	- Not clear why this is the case\\
	- Body composition differences between gender and increasing BMI\\
	- Do females and obese subjects need different thresholds as suggested by some authors?

\subsection{OJ vs biochemistry}
	Jaundice associated with systemic inflammation, anaemia and electrolyte abnormalities 

\subsection{ $\dot{V}_{O_2}$AT associated with systemic inflammation}
	Sultan et al: NLR associated with low VO2AT in colorectal patients
	
\subsection{Peak $\dot{V}_{O_2}$ was related to hemoglobin - p92}
	- Heart failure patients\\
	- Healthy volunteers after blood transfusion\\
	- Blood loss associated with adverse outcomes\\
	- Is there a role for preoperative optimisation?
	
\subsection{$\dot{V}_{O_2}$ vs cancer}
- Does sarcopenia play a role?\\
- Does the metabolic effects of cancer including inflammation, poor nutrition, etc. impair aerobic capacity?

\clearpage

\section{Chapter 4}

\subsection{BC vs clinicopathological char T4.1 p106}

- Females had more subcutn fat, men had more visceral fat and skeltal muscle\\
- Lower $\dot{V}_{O_2}$AT and $\dot{V}_{O_2}$peak -- more subcutn fat but not visceral fat\\
- Lower $\dot{V}_{O_2}$peak -- lower skeletal muscle\\
- Lower Hb -- lower skeletal muscle

\subsection{BC vs sex/bmi - F4.2 p107}

- As BMI increases, body composition differences between men and women becomes apparent.\\
- The proportion of skeletal muscle in comparison to fat is low in women (32 vs 38) even at normal BMI. \\
- This proportion falls even further in obese women to 14\% while obese men have 22\% muscle contributing to their weight\\
- Obese men and obese women are very different in their body composition

\subsection{BC vs CPET - T4.2 p110}

- All calculations were controlled for the effect of gender\\
- Pulmonary function tests were controlled for effect of gender and age\\
- FVC/FEV positively correlated with skeletal muscle but not fat\\
- Differentiate between absolute and corrected $\dot{V}_{O_2}$.\\
- Differentiate between the 3 phases of exercise\\
- Positive correlation between skeletal muscle and load, tidal volume, abs $\dot{V}_{O_2}$, o2pulse\\
- Negative correlation between corrected $\dot{V}_{O_2}$ and fat - why is this?

\subsection{Scatter plots - f4.4 p111}

- solid line is absolute $\dot{V}_{O_2}$ before normalising for weight\\
- dotted line is corrected $\dot{V}_{O_2}$ after normalising for weight\\
- the negative correlation that appears with adipose tissue is 'spurious'

\subsection{Scaling for other factors t4.3 p113, Fig. 4.5 p114}

- the ideal scaling factor should remove any correlation with body habitus\\
- weight, height squared, BMI, skeletal muscle and eLBM were used\\
- Correction by eLBM resulted in loss of all correlations and probably best

\clearpage

\section{Chapter 5}
\subsection{Preop inflammation vs Postop CRP}
- Preop CRP and albumin affected the inflammatory response in the first week\\
- This is best seen in Fig 5.3 and Table 5.7 on p142\\
- Line 0: absence of a preop SIRS; Line 1: only CRP abnormal;\\
- Line 2: only Albumin abnormal; Line 3:both are abnormal\\
- Postop CRP response was elevated if preop CRP was high with normal albumin\\
- Hypoalbuminemia resulted in a dampened inflammatory response

\subsection{OJ vs post op inflammation}
- Postop CRP significantly lower in jaundiced patients\\
- Preop CRP greater in OJ. Hence this shows a reversal of this phenomenon\\
- Postop albumin lower- but this was a continuation of preop hypo-albuminemia

\subsection{CPET, comorbidity, biochemistry, etc. vs postop inflammation}
- No relationship with aerobic capacity, comorbidity and other factors\\
- Probably due to dominant effect of preop inflammation and OJ\\
- Low alb associated with multiple abnormalities as a continuum from preop

\subsection{BC vs postop inflammation}
- No association with CRP/neutrophil count\\
- Low albumin $\rightarrow$ low skeletal muscle

\subsection{Discussion}
- Preop systemic inflammation affects postop SIRS\\
- When hypoalbuminemia is present, effect of CRP is lost and postop SIRS is dampened\\
- hypoalbuminemia is a hallmark of CARS\\
- Preop jaundice affects postop SIRS\\
- Immune dysregulation/impaired CRP response are seen with liver dysfunction\\
- Bernal 2014: Liver transplantation - hypoalbuminemia - low $\dot{V}_{O_2}$\\
- Richards 2012: prep inflammation - low  skeletal muscle\\
- Hassen 2007: low skeletal muscle - greater postop inflammation after AAA\\
- Sarcopenic-obesity $\rightarrow$ inflammation

\clearpage

\section{Chapter 6}

\subsection{Numbers t6.4 p165}
- Total 188; POPF 62; Infections 84
\subsection{Infective complications, preop factors and other outcomes}
- $\dot{V}_{O_2}$AT $\rightarrow$ infective complications\\
- infective complications $\rightarrow$ other complications, critical care, mortality

\subsection{CRP vs POPF p163}
- CRP does not predict severity of POPF; stress on importance of drain amylase

\subsection{CRP vs infective complications p 165}
- Useful in the absence of POPF\\
- POPF triggers other complications and early SIRS no longer a major factor

\subsection{Albumin, WCC}
- Not particularly useful clinically in spite of some associations later in the week

\subsection{ROC analysis p171}
- D3 CRP 178 NPV 0.79; D4 CRP 125 NPV 0.83

\subsection{D4 CRP and Length of Stay t6.9 p172}
- 2/3rds stayed less than 14 days\\
- Nearly 90\% left critical care in under 1 week\\
- Only 7.7\% were readmited to crit care against 21\% if CRP over 125\\
- These were in the absence of enhanced recovery or CRP being included in discharge criteria

\subsection{Discussion}
- POPF $\rightarrow$ outcomes; ISGPF\\
- CARS $\rightarrow$ SIRS;D4 CRP $\rightarrow$ surgical trauma 





\end{document}