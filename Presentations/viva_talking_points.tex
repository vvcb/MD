\documentclass[12pt,a4paper]{article}
\usepackage[utf8]{inputenc}
\usepackage{amsmath}
\usepackage{amsfonts}
\usepackage{amssymb}
\usepackage{graphicx}
\author{Vishnu V Chandrabalan}

\renewcommand{\familydefault}{\sfdefault}

\begin{document}

\section{Chapter 3}

\subsection{OJ vs $\dot{V}_{O_2}$}
	- OJ was thought to be associated with adverse haemodynamic events

	- Routine PBD was advised based on this

	- Multiple studies have reported adverse events after PBD

	- Management of Bil>250 is unclear

	- Here, OJ not associated with VO2

	- Parker et al, Junejo et al

	- \textbf{NO JUSTIFICATION FOR PBD}

	-Figures on P87, Tables p91, 92

\subsection{VO2 vs Female/BMI}
	- Not clear why this is the case

	- Body composition differences between gender and increasing BMI

	- Do females and obese subjects need different thresholds as suggested by some authors?

\subsection{OJ vs biochemistry}
	Jaundice associated with systemic inflammation, anaemia and electrolyte abnormalities 

\subsection{ VO2AT associated with systemic inflammation}
	Sultan et al: NLR associated with low VO2AT in colorectal patients
	
\subsection{Peak Vo2 was related to hemoglobin - p92}
	- Heart failure patients

	- Healthy volunteers after blood transfusion

	- Blood loss associated with adverse outcomes

	- Is there a role for preoperative optimisation?
	
\subsection{vo2 vs cancer}
- Does sarcopenia play a role?

- Does the metabolic effects of cancer including inflammation, poor nutrition, etc. impair aerobic capacity?

\clearpage

\section{Chapter 4}

\subsection{BC vs clinicopathological char T4.1 p106}

- Females had more subcutn fat, men had more visceral fat and skeltal muscle

- Lower $\dot{V}_{O_2}$AT and $\dot{V}_{O_2}$peak -- more subcutn fat but not visceral fat

- Lower $\dot{V}_{O_2}$peak -- lower skeletal muscle

- Lower Hb -- lower skeletal muscle

\subsection{BC vs sex/bmi - F4.2 p107}

- As BMI increases, body composition differences between men and women becomes apparent.

- The proportion of skeletal muscle in comparison to fat is low in women (32 vs 38) even at normal BMI. 

- This proportion falls even further in obese women to 14\% while obese men have 22\% muscle contributing to their weight

- Obese men and obese women are very different in their body composition

\subsection{BC vs CPET - T4.2 p110}

- All calculations were controlled for the effect of gender

- Pulmonary function tests were controlled for effect of gender and age

- FVC/FEV positively correlated with skeletal muscle but not fat

- Differentiate between absolute and corrected $\dot{V}_{O_2}$.

- Differentiate between the 3 phases of exercise

- Positive correlation between skeletal muscle and load, tidal volume, abs $\dot{V}_{O_2}$, o2pulse

- Negative correlation between corrected $\dot{V}_{O_2}$ and fat - why is this?

\subsection{Scatter plots - f4.4 p111}

- solid line is absolute $\dot{V}_{O_2}$ before normalising for weight

- dotted line is corrected $\dot{V}_{O_2}$ after normalising for weight

- the negative correlation that appears with adipose tissue is 'spurious'

\subsection{Scaling for other factors t4.3 p113, Fig. 4.5 p114}

- the ideal scaling factor should remove any correlation with body habitus

- weight, height squared, BMI, skeletal muscle and eLBM were used

- Correction by eLBM resulted in loss of all correlations and probably best


\clearpage

\end{document}