% Chapter 01 - Introduction

\chapter*{An investigation of the clinical utility of preoperative cardiopulmonary exercise testing in patients undergoing major pancreatic surgery.}

\lhead{\emph{Response to examiners' comments}} % This is for the header on each page - perhaps a shortened title

\vfill

\hrule
\vfill
{\LARGE \textbf{Response to examiners' comments}}
\vfill
\hrule

\vfill

\textbf{Mr. Vishnu V Chandrabalan}

\textbf{University of Department of Surgery}

\textbf{University of Glasgow}

\vfill

\clearpage
%----------------------------------------------------------------------------------------
Dear Sir,
	
I am grateful for the opportunity to respond to the examiners' comments. 
I have considered the examiners' comments carefully and the resulting changes have improved the thesis. 
Please find my responses to the comments below.
	
\textbf{Chapter 1 - More discussion of key RCTs}

	Point taken.	
	The following paragraphs have been included under Section 1.4.4 in Chapter 1 and discusses some of the key randomised trials of operative strategies aimed at reducing postoperative complications after pancreaticoduodenectomy.
	
	\textquotedblleft
	Several studies have attempted to evaluate various surgical strategies in improving perioperative outcomes after pancreaticoduodenectomy. 
	A prospective, multi-center, randomised controlled trial of 170 patients undergoing pancreaticoduodenectomy did not demonstrate any significant difference between a Whipple's classical pancreaticoduodenectomy  and pylorus preserving pancreaticoduodenectomy \parencite{tran_pylorus_2004}. 
	A Cochrane review in 2008 reported on 7 randomised controlled trials comparing classical pancreaticoduodenectomy with pylorus-preserving pancreaticoduodenectomy  and found no difference in morbidity, mortality or survival \parencite{diener_pancreaticoduodenectomy_2008}. 
	
	A prospective randomised trial comparing pancreaticogastrostomy versus pancreaticojejunostomy in patients undergoing pancreaticoduodenectomy reported that there was no difference in the incidence of overall complications or pancreatic fistula. 
	However, patients who had a pancreaticogastrostomy had a significantly lower rate of biliary fistula, postoperative collections, delayed gastric emptying as well as lower incidence of multiple surgical complications \parencite{bassi_reconstruction_2005}. 
	A meta-analysis of eleven studies (one randomised trial, 2 non-randomised prospective trials and eight observational studies) appeared to show improved outcomes in patients undergoing pancreaticogastrostomy rather than a pancreaticojejunostomy \parencite{mckay_meta-analysis_2006}. 
	However, an earlier trial by Yeo and co-workers failed to demonstrate any difference in outcomes between either technique \parencite{yeo_prospective_1995}.
	
	In spite of a lack of any clear evidence of efficacy, octreotide continues to be used in many centres including ours to reduce the incidence of postoperative pancreatic fistula. 
	While some early randomised trials reported the benefits of octreotide in decreasing the risk of postoperative pancreatic fistula \parencite{montorsi_efficacy_1995,nakatsuka_octreotide_2000}, other more recent trials have not reproduced these results \parencite{lowy_prospective_1997, yeo_does_2000, kollmar_prophylactic_2008}. 
	
	Surgical drains after pancreaticoduodenectomy have been a subject of considerable debate. 
	A prospective randomised controlled trial of 179 patients in 2001 reported that intra-abdominal drains placed at the time of surgery did not reduce the incidence of complications \parencite{conlon_prospective_2001}.
	
	Two randomised trials comparing early versus delayed drain removal in patients with low risk of pancreatic fistula after pancreaticoduodenectomy reported that early drain removal was associated with lower incidence of pancreatic fistula, abdominal complications and pulmonary complications \parencite{kawai_early_2006,bassi_early_2010}. 
	
	However, these pancreas-specific risk factors are not modifiable before surgery, have been studied extensively before and are not the subject of this thesis.
	\textquotedblright
	
	\textquotedblleft		
	The following has been added to Section 1.3 in Chapter 1 and adds to the already detailed discussion of some of the key randomised trials in adjuvant therapy after pancreaticoduodenectomy.
	
	Early randomised trials reported improved survival in patients with inoperable pancreatic cancer \parencite{mallinson_chemotherapy_1980} as well as in patients who had undergone potentially curative surgery \parencite{bakkevold_adjuvant_1993}. 
	
	Oettle and co-workers reported on the results of a European multi-center, randomised controlled phase 3 trial comparing the results of surgery verus surgery and adjuvant chemotherapy with 6 cycles of gemcitabine in patients with pancreatic ductal adenocarcinoma \parencite{oettle_adjuvant_2007}. 
	The median disease free survival in patient who received gemcitabine chemotherapy was 13.4 months while it was only 6.9 months in the surgery-only group. 
	However, there was no difference in overall survival.
	
	Increasingly, neo-adjuvant therapies are being used in patients with initially unresectable or 'borderline-resectable' cancers in an attempt to improve resectability. 
	In a meta-analysis of 111 studies, Gillen and co-workers reported that approximately one-third of initially unresectable patients can be expected to have resectable tumours after neo-adjuvant treatment \parencite{gillen_preoperative/neoadjuvant_2010}. 
	The authors recommended that such patients should be actively included in neo-adjuvant protocols and re-staged for resection after treatment.
	\textquotedblright
		
\textbf{Chapter 1 - Short discussion of aetiology needed}
	
	Point taken. The following subsection titled 'Aetiology of pancreatic cancer' has been included in Chapter 1 under Section 1.1.
	
	\textquotedblleft
	Several factors have been identified as being associated with an increased risk of pancreatic cancer.
	The most important risk factor associated with increased incidence of pancreatic cancer is tobacco. 
	Tobacco smoking has been reported to be associated with at least a 2-fold increased risk of pancreatic cancer and this risk increases to 5-fold in patients with over 40 pack-years of smoking \parencite{raimondi_early_2007, iodice_tobacco_2008}. 
	This increased risk persists for at least 10 years after cessation of smoking \parencite{iodice_tobacco_2008}.
	
	The association between alcohol consumption and pancreatic cancer is less clear. 
	While alcohol consumption on its own has not been shown to result in an increased risk of pancreatic cancer, the incidence of pancreatic cancer in heavy drinkers is greater than in the general population.
	This may be due to the confounding effect of cigarette smoking in this cohort of patients \parencite{jiao_alcohol_2009, rohrmann_ethanol_2009}.
	
	Obesity and high caloric intake have also been reported to be associated with increased risk of pancreatic cancer although the underlying mechanisms and the effects of confounding factors are poorly understood \parencite{berrington_de_gonzalez_meta-analysis_2003, larsson_body_2007, li_body_2009}. 
	
	Other possible risk factors include increasing age, diet high in saturated fats, chronic pancreatitis, family history of pancreatic cancer and genetic abnormalities resulting in cancer syndromes \parencite{raimondi_epidemiology_2009, maisonneuve_epidemiology_2010}.
	It is of note that some of the important risk factors such as smoking, obesity, poor diet and increasing age may also be associated with other medical comorbidities including cardiorespiratory diseases that may have a significant impact on patient fitness.
	\textquotedblright
	
\textbf{Chapter 2 - Review patient numbers on page 63}
	
	Point taken. Patient numbers have been updated to account for the missing information and this section now reads as below accounting for all the 100 patients included in this study.
	
	\textquotedblleft
	Pathological examination of the resected specimen showed pancreatic ductal adenocarcinoma (n=37), ampullary adenocarcinoma (n=18), cholangiocarcinoma (n=17), duodenal adenocarcinoma (n=6), intra-ductal papillary mucinous neoplasia (n=4), neuroendocrine tumours (n=7), other neoplasia (n=6) or chronic pancreatitis (n=5).
	\textquotedblright
	
\textbf{Chapter 2 - Aim and end points must match }
	
	Point taken. The aims of this chapter have been updated as below to match the end-points described later in methods and results sections. 
	
	\textquotedblleft
	The aim of the present study was to evaluate the role of various measures of patient physiological fitness including cardiopulmonary exercise testing in predicting postoperative length of stay, major postoperative adverse events including operative mortality and fitness to undergo adjuvant therapy when indicated after pancreaticoduodenectomy.
	\textquotedblright
	
\textbf{Page 69 - Error in AT numbers}
	
	The results presented in this table have been re-analysed using the primary database and have been found to be accurate. Therefore, no changes have been made.
	
\textbf{Page 72 - Remove unjustified/unsupported conclusion, Page 73 - Remove GPS conclusion}
	
	Point taken. The following sentence has been \underline{\textit{removed}} from the discussion in Chapter 2.
	
	\textquotedblleft
	Therefore, it would appear that objective measurement of patient physiological fitness using cardiopulmonary exercise testing is superior to conventional measures of patient fitness including the POSSUM Physiology Score or the modified Glasgow Prognostic Score and may have a role in predicting short-term outcome which in turn affects the overall management of these patients including receipt of adjuvant therapy.
	\textquotedblright
	
\textbf{Chapter 3, page 96 - Remove conclusions not based on thesis}
	
	Point taken. The following sentence has been \underline{\textit{removed}} from the discussion in Chapter 3.
	
	\textquotedblleft
	Taken together, these results appear to suggest that preoperative biliary drainage and resolution of jaundice are unlikely to improve aerobic capacity and cardiopulmonary physiology in these patients.
	\textquotedblright
	
\textbf{Chapter 5,6 - Reduce to more understandable key issues. Several irrelevant tables should be removed.}

	Point taken. 
	Chapter 5 has been restructured and simplified as follows. 
	Postoperative CRP alone has been used as a marker of postoperative systemic inflammation and postoperative albumin and neutrophil counts have been ignored.
	All tables and figures referring to postoperative albumin and postoperative neutrophil counts have been removed. 
	The text has been modified to reflect these changes. 
	All figures and tables have been updated with a brief description of their contents. 
	This has resulted in a substantially clearer and more understandable chapter that focuses on the relationship between preoperative patient factors and the postoperative CRP response.
	
	Chapter 6 has been revised and simplified as follows.
	The chapter has been rewritten to focus on the role of postoperative CRP in predicting complications after pancreaticoduodenectomy.
	All references to postoperative albumin, white cell count and neutrophil counts have been removed.
	All redundant/irrelevant tables and figures that relate to postoperative white cell count, neutrophil count and albumin have been removed.
	This has resulted in a substantially clearer and more concise chapter that emphasises the importance of early postoperative CRP in predicting infective complications after pancreaticoduodenectomy.
	
\textbf{All acronyms/abbreviations on list}
	
	A list of abbreviations has been included.
	
\textbf{All tables should have units (e.g. mg/L, etc) and legends.}
	
	Point taken. 
	All tables have been updated to include units. 
	Legends have been added to all figures and tables providing a brief description of the results and the statistical tests used.