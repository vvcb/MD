%OJ - PEAK EXERCISE
\begin{sidewaystable}[p]
	\caption{The relationship  between obstructive jaundice and cardiopulmonary exercise test parameters at peak exercise in patients undergoing pancreaticoduodenectomy.  }
	\label{table:cpet_oj_peak}
	\centering
	\renewcommand{\arraystretch}{1.4} %Increases space between rows
	%\setlength{\tabcolsep}{9pt} %sets the space between columns
		%6 columns   
	\begin{tabular}{|l| c c c c c|}
		\hline
		                               &         \multicolumn{4}{c}{Preoperative Serum Bilirubin ($\mu$mol/L)}          &  \\
		Peak Exercise                  & $\leq$ 17         & 18-35             & 35-250            & $>$ 250            & \textit{p} \\ \hline
		Load  (Watts)                  & 94.0 (76.5-114.5) & 87.5 (56.0-107.0) & 73.0 (58.0-108.0) & 85.0 (66.0-101.0)  & 0.066      \\
		Min. Ventilation (l/min)       & 53.5 (46.0-69.0)  & 46.5 (34.0-62.0)  & 46.0 (38.0-63.0)  & 48.0 (37.0-67.0)   & 0.088      \\
		Tidal Volume (litres)          & 1.95 (1.6-2.41)   & 1.64 (1.46-1.98)  & 1.62 (1.35-2.19)  & 1.86 (1.18-2.21)   & 0.028      \\
		$\dot{V}_{O_2}$ (litres/min)   & 1.33 (1.09-1.57)  & 1.14 (0.90-1.32)  & 1.08 (0.85-1.5)   & 1.11 (0.89-1.38)   & 0.007      \\
		$\dot{V}_{O_2}$/kg (ml/kg/min) & 17.2 (14.45-22.0) & 14.7 (13.5-17.3)  & 15.1 (12.7-19.9)  & 15.7 (13.3-19.2)   & 0.056      \\
		$\dot{V}_E/\dot{V}_{O_2}$      & 40.5 (36.2-46.8)  & 43.1 (37.3-46.5)  & 41.9 (38.9-47.4)  & 46.7 (42.1-55.2)   & 0.073      \\
		$\dot{V}_{CO_2}$ (litres/min)  & 1.67 (1.37-2.01)  & 1.32 (1.02-1.84)  & 1.29 (1.04-1.88)  & 1.47 (1.03-1.76)   & 0.016      \\
		$\dot{V}_E/\dot{V}_{CO_2}$     & 31.9 (29.5-34.9)  & 32.1 (29.4-36.4)  & 33.1 (30.2-37.8)  & 37.4 (30.9-40.8)   & 0.110      \\
		RER                            & 1.28 (1.20-1.42)  & 1.27 (1.20-1.42)  & 1.28 (1.22-1.36)  & 1.36 (1.25-1.42)   & 0.675      \\
		${PET_O}_2$ (mmHg)             & 121 (118.5-125)   & 122 (120-126)     & 122 (120-125)     & 126 (123-128)      & 0.026      \\
		${PET_{CO}}_2$ (mmHg)          & 35 (32.5-38)      & 35 (32-38)        & 34 (30-38)        & 33 (29-36)         & 0.283      \\
		$O_2$Pulse (ml/beat)           & 11.9 (9.11-14)    & 10.0 (8.0-12.0)   & 11.0 (9.61-13.93) & 11.65 (8.26-13.94) & 0.132      \\
		Heart rate (/min)              & 140 (125-158)     & 140 (129-152)     & 129 (114-144)     & 134 (128-158)      & 0.158      \\
		Respiratory Rate (/min)        & 30 (26-34)        & 32 (28-35)        & 30 (26-34)        & 31 (28-36)         & 0.512      \\
		Exercise Duration (minutes)    & 8.2 (6.2-10.4)    & 6.9 (5.5-9.2)     & 6.6 (4.8-9.0)     & 8.2 (5.2-9.7)      & 0.164      \\ \hline
		\multicolumn{6}{l}{Values are median (inter-quartile range); \textit{p} - Kruskal-Wallis test}
	\end{tabular}
		\medskip
		\begin{flushleft}
			At peak exercise, there were several statistically significant but non-linear relationships between obstructive jaundice and CPET parameters. However, total exercise duration, exercise load achieved and $\dot{V}_{O_2}$Peak (ml/kg/min) were not different between the non-jaundiced, jaundiced and severely jaundiced groups. \\
			$\dot{V}_{O_2}$ - Oxygen consumption, $\dot{V}_{CO_2}$ - Exhaled $CO_2$, PET$O_2$/$CO_2$ - Partial pressure of end-tidal $O_2$/$CO_2$, $O_2$Pulse - Oxygen pulse.
		\end{flushleft}
\end{sidewaystable}
