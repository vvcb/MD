%OJ - PFT
\begin{sidewaystable}[p]
	\caption{The relationship  between obstructive jaundice and pulmonary function tests in patients undergoing pancreaticoduodenectomy.}
	\label{table:cpet_oj_pft}
	\centering
	\renewcommand{\arraystretch}{1.4} %Increases space between rows
	\setlength{\tabcolsep}{9pt} %sets the space between columns
	
	\begin{tabular}{|l| c c c c c|}
		\hline
		                        &       \multicolumn{4}{c}{Preoperative Serum Bilirubin ($\mu$mol/L)}        &  \\
		Pulmonary Function Test & $\leq$ 17        & 18-35            & 35-250            & $>$ 250          & \textit{p} \\ \hline
		FVC (litres)            & 4.09 (3.49-4.69) & 3.76 (3.38-4.59) & 3.76 (3.16-4.05)  & 3.35 (2.85-4.38) & 0.092      \\
		FEV1 (litres)           & 2.95 (2.39-3.51) & 2.90 (2.12-3.34) & 2.68 (2.37-3.07)  & 2.72 (2.2-3.28)  & 0.556      \\
		Predicted FEV1 (\%)     & 105.0 (91-116)   & 98.50 (87-114)   & 103.0 (95-111.5)  & 101.0 (94-116)   & 0.761      \\
		FEV1/FVC                & 72.0 (65-77)     & 73.0 (65-78)     & 75.50 (70.5-79.5) & 78.0 (70-82)     & 0.115      \\
		Predicted FEV1/FVC (\%) & 94.0 (87-102)    & 96.0 (86-100)    & 99.0 (93-103)     & 102.0 (88-108)   & 0.107      \\ \hline
		\multicolumn{6}{l}{Values are median (inter-quartile range); \textit{p} - Kruskal-Wallis test}
	\end{tabular}
	\medskip
	\caption*{Pulmonary function tests performed immediately prior to cardiopulmonary exercise testing were compared between the non-jaundiced, jaundiced and severely jaundiced patients (Total n = 138). This did not show any relationship between obstructive jaundice and preoperative pulmonary function tests. FVC - Forced Vital Capacity, FEV1 - Forced Expiratory Volume in 1 second.}
\end{sidewaystable}



