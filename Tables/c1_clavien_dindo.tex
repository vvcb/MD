\begin{sidewaystable}[htbp]
	\centering
	\caption{The Clavien-Dindo classification of surgical complications.}
	\label{table:clavien-dindo}	
	\renewcommand{\arraystretch}{1.7} %Increases space between rows
	\setlength{\tabcolsep}{14pt} %sets the space between columns
	\begin{tabular}{|l m{15cm}|}
		\hline
		Grade       & Description \\ \hline
		Grade I     & Any deviation from the normal postoperative course without the need for pharmacological treatment or surgical, endoscopic and radiological interventions.  \\
		Grade II    & Requiring pharmacological treatment with drugs other than such allowed for grade I complications. Blood transfusions and total parenteral nutrition are also included.  \\
		Grade III   & Requiring surgical, endoscopic or radiological intervention  \\
		            & Grade III-a: - intervention not under general anaesthesia \\
		            & Grade III-b: - intervention under general anaesthesia  \\
		Grade IV    & Life-threatening complication (including CNS complications)‡ requiring  HDU/ICU-management  \\
		            & Grade IV-a: - single organ dysfunction (including dialysis)     \\
		            & Grade IV-b: - multi organ dysfunction \\
		Grade V     & Death of a patient  \\ \hline
		Suffix 'd': & If the patients suffers from a complication at the time of discharge,  the suffix  “d”  (for ‘disability’) is added to the respective grade of complication. This label indicates the need for a follow-up to fully evaluate the complication. \\ \hline
	\end{tabular}
\end{sidewaystable}



 	
