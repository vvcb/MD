\documentclass[12pt,a4paper]{article}
\usepackage[utf8]{inputenc}
%\usepackage{amsmath}
%\usepackage{amsfonts}
%\usepackage{amssymb}
%\usepackage{graphicx}
\author{Vishnu V Chandrabalan}
\begin{document}

\textbf{Response to Reviewer’s comments}
	
\textbf{An investigation of the clinical utility of preoperative cardiopulmonary exercise testing in patients undergoing major pancreatic surgery.}
	
Dear Sir,
	
I am grateful for the opportunity to respond to the examiners' comments. 
I have considered the examiners' comments carefully and these have improved the thesis. 
Please find my responses to the comments below.
	
\textbf{Chapter 1 - More discussion of key RCTs}

	Point taken.	
	The following paragraph has been included under Section 1.4.4 in Chapter 1 and discusses some of the key randomised trials of operative strategies aimed at reducing postoperative complications after pancreaticoduodenectomy.
	
	Several studies have attempted to evaluate various surgical strategies in improving perioperative outcomes after pancreaticoduodenectomy. 
	A Cochrane review in 2008 reported on 7 randomised controlled trials comparing classical pancreaticoduodenectomy with pylorus-preserving pancreaticoduodenectomy  and found no difference in morbidity, mortality or survival. 
	A prospective randomised trial comparing pancreaticogastrostomy versus pancreaticojejunostomy in patients undergoing pancreaticoduodenectomy reported that there was no difference in the incidence of overall complications or pancreatic fistula. 
	However, patients who had a pancreaticogastrostomy had a significantly lower rate of biliary fistula, postoperative collections, delayed gastric emptying as well as lower incidence of multiple surgical complications. 
	Surgical drains after pancreaticoduodenectomy have been a subject of considerable debate. 
	A randomised trial comparing early versus delayed drain removal in patients with low risk of pancreatic fistula after pancreaticoduodenectomy reported that early drain removal was associated with lower incidence of pancreatic fistula, abdominal complications and pulmonary complications. 
	Hospital stay and costs were also lower.
	
	The following has been added to Section 1.3 in Chapter 1 and adds to the already detailed discussion of some of the key randomised trials in adjuvant therapy after pancreaticoduodenectomy.
	
	Oettle and co-workers reported on the results of a European multi-center, randomised controlled phase 3 trial comparing the results of surgery verus surgery and adjuvant chemotherapy with 6 cycles of gemcitabine in patients with pancreatic ductal adenocarcinoma. 
	The median disease free survival in patient who received gemcitabine chemotherapy was 13.4 months while it was only 6.9 months in the surgery-only group. However, there was no difference in overall survival.
	
	
\textbf{Chapter 1 - Short discussion of aetiology needed}
	
	Point taken. The following subsection titled 'Aetiology of pancreatic cancer' has been included in Chapter 1 under Section 1.1.
	
	Several factors have been identified as being associated with an increased risk of pancreatic cancer. 
	The most important risk factor associated with increased incidence of pancreatic cancer is tobacco. 
	Obesity has also been reported to be associated with increased risk of pancreatic cancer although the underlying mechanisms are unclear. 
	Other possible risk factors include increasing age, diet high in saturated fats, chronic pancreatitis, family history of pancreatic cancer and genetic abnormalities resulting in cancer syndromes. 
	It is of note that some of the important risk factors such as smoking, obesity, poor diet and increasing age may also be associated with other medical comorbidities including cardiorespiratory diseases that may have a significant impact on patient fitness.
	
\textbf{Chapter 2 - Review patient numbers on page 63}
	
	Point taken. Patient numbers have been updated to account for the missing information and this section now reads as below accounting for all the 100 patients included in this study.
	
	Pathological examination of the resected specimen showed pancreatic ductal adenocarcinoma (n=37), ampullary adenocarcinoma (n=18), cholangiocarcinoma (n=17), duodenal adenocarcinoma (n=6), intra-ductal papillary mucinous neoplasia (n=4), neuroendocrine tumours (n=7), other neoplasia (n=6) or chronic pancreatitis (n=5).
	
\textbf{Chapter 2 - Aim and end points must match }
	
	Point taken. The aims of this chapter have been updated as below to match the end-points described later in methods and results sections. 
	
	The aim of the present study was to evaluate the role of various measures of patient physiological fitness including cardiopulmonary exercise testing in predicting postoperative length of stay, major postoperative adverse events including operative mortality and fitness to undergo adjuvant therapy when indicated after pancreaticoduodenectomy.
	
\textbf{Page 69 - Error in AT numbers}
	
	The results presented in this table have been re-analysed using the primary database and have been found to be accurate. Therefore, no changes have been made.
	
\textbf{Page 72 - Remove unjustified/unsupported conclusion, Page 73 - Remove GPS conclusion}
	
	Point taken. The following sentence has been removed from the discussion in Chapter 2.
	
	Therefore, it would appear that objective measurement of patient physiological fitness using cardiopulmonary exercise testing is superior to conventional measures of patient fitness including the POSSUM Physiology Score or the modified Glasgow Prognostic Score and may have a role in predicting short-term outcome which in turn affects the overall management of these patients including receipt of adjuvant therapy.
	
\textbf{Chapter 3, page 96 - Remove conclusions not based on thesis}
	
	Point taken. The following sentence has been removed from the discussion in Chapter 3.
	
	Taken together, these results appear to suggest that preoperative biliary drainage and resolution of jaundice are unlikely to improve aerobic capacity and cardiopulmonary physiology in these patients.
	
\textbf{Chapter 5,6 - Reduce to more understandable key issues. Several irrelevant tables should be removed.}

	Can we talk about this over the phone, please?
	
\textbf{All acronyms/abbreviations on list}
	
	A list of abbreviations has been included.
	
\textbf{All tables should have units (e.g. mg/L, etc) and legends.}
	
	Point taken. 
	All tables have been updated to include units. 
	Legends have been added to all figures and tables providing a brief description of the results
	depicted.
	
\textbf{All values and ranges need defining.}
	
	Are they looking for normal ranges for the various parameters like CRP,albumin, etc.?
\end{document}